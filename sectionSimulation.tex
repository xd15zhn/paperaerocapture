\section{仿真}
被控对象的输入(控制量)为倾侧角$\sigma$,
输出为位置$\vec{r}$。
可以用上一节中推导出的式\eqref{eqModelTotal}建立被控对象微分方程,
也可以建立向量形式的微分方程
\[\ddot{\vec{r}} = \frac{\mu}{|\vec{r}|^3}\vec{r}+\vec{L}+\vec{D}\]
其中阻力向量为$\vec{D} = -D\frac{\vec{v}}{|\vec{v}|}$,
升力向量$\vec{L}$与速度向量$\vec{v}$垂直,
定义一个垂直于地面的法向量$\vec{n}=[0,0,1]^\text{T}$,
构造一个与$\vec{n}$和$\vec{v}$都垂直的单位向量$\vec{n}_2$,
和另一个与$\vec{n}_2$和$\vec{v}$都垂直且方向向上的单位向量$\vec{n}_1$,
则$\vec{n}_1$位于向量$\vec{v}$和$\vec{n}$张成的平面内,
此时$\vec{L}$与$\vec{n}_1$的夹角即为倾侧角,
满足
\begin{align*}
    \vec{n}_2 =& \vec{v}\times\vec{n} \\
    \vec{n}_1 =& \times\vec{n}_2\times\vec{v}
\end{align*}
则
\[\vec{L}=\vec{n}_1\cos\sigma + \vec{n}_2\sin\sigma\]
