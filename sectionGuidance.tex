\section{预测校正制导}
理论上当大气入口条件和变轨目标确定后,
即可得出一条确定轨迹,气动飞行期间不用施加控制即可实现既定的捕获任务。
但实际飞行中存在极强的环境不确定性,飞行器及进入轨道本身也存在初始条件的不确定性,因此需要设计在轨制导律。
气动飞行的制导方法主要分为标称轨迹制导和预测校正制导。
制导算法的鲁棒性也是需要重点考虑的问题关键,在不确定性强的环境下,预测校正制导相对标称轨迹的解析制导更具备优势。
另一方面,气动捕获整个过程一般持续数百秒,不能依靠地面控制,只能选择在轨自主控制方法,这就对制导律的计算复杂度提出了要求\cite{dqingyuan2019}。
