%%%%%%%%%%%%%%%%%%%%%%%%%%%%%%%%%%%%%%%%%%%%%%%%%%%%%%%%%%%%%%%%
% 节
%%%%%%%%%%%%%%%%%%%%%%%%%%%%%%%%%%%%%%%%%%%%%%%%%%%%%%%%%%%%%%%%
\section{引\quad 言}


%%%%%%%%%%%%%%%%%%%%%%%%%%%%%%%%%%%%%%%%%%%%%%%%%%%%%%%%%%%%%%%%
% 节
%%%%%%%%%%%%%%%%%%%%%%%%%%%%%%%%%%%%%%%%%%%%%%%%%%%%%%%%%%%%%%%%
\section{实时仿真器简介}
宇宙中各天体的自转或公转周期通常以天甚至以年计算,且轨道速度快,空间尺度大。为了能够在计算机上直观地实时展示仿真结果,需要对一些公式的单位进行换算。\par
万有引力公式中,
$$\frac{\text{d}^2\vec{r}}{\text{d}t^2}=-\frac{\mu}{r^3}\vec{r}$$
中,$r$的单位是km,$t$的单位是s。因为实际的尺度太大,一个天文单位达到了$10^8$数量级,为了便于仿真,需要将实际尺度中的时间和距离变换到一个合适的时空坐标系下。
首先列举一些名词解释。实际时间:指宇宙尺度上的实际时间;无单位求解器时间:指微分方程求解器中自变量的值,没有单位,简称求解器时间,对应的速度和距离也都没有单位;仿真器消耗时间:例如仿真器exe文件运行1秒对应实际时间的1000秒,此处的1秒就是仿真器消耗时间,简称仿真器时间。现设置求解器时间的1单位时间等于实际时间的$10^4$秒,1单位距离等于实际距离$10^6$km,即在仿真软件中,
$$t_1=10^{-4}t,\ r_1=10^{-6}r$$
其中$t$和$r$分别为以秒和千米为单位的实际时间和实际距离,而$t_1$和$r_1$分别表示求解器时间和求解器距离。举个例子,1个天文单位为$r=1.5*10^8$km,则仿真软件中的1个天文单位为$r_1=150$。
然后可以计算出实际时间/速度/距离和求解器时间/速度/距离之间的一些换算关系
$$\begin{aligned}
v_1 =& \frac{\text{d}r_1}{\text{d}t_1}
 = \frac{10^{-6}\text{d}r}{10^{-4}\text{d}t} = 10^{-2}v \\
a_1 =& \frac{\text{d}}{\text{d}t_1}\frac{\text{d}r_1}{\text{d}t_1}
 = 10^2\frac{\text{d}^2r}{\text{d}t^2} \\
\mu_1(\frac{(10^{-6}\text{km})^3}{\text{kg}\cdot(10^{-4}\text{s})^2})
 =& 10^{-10}\mu(\frac{\text{km}^3}{\text{kg}\cdot \text{s}^2}) \\
-\frac{\mu_1}{r_1^2} =& -10^2\frac{\mu}{r^2} = 10^2\frac{\text{d}^2r}{\text{d}t^2} = a_1
\end{aligned}$$
可以验证,实际中的微分方程
$$a=-\frac{\mu}{r^2}$$
变换到仿真软件中仍然为
$$a_1=-\frac{\mu_1}{r_1^2}$$
若仿真步长设置为0.01求解器时间,每帧仿真10步,则实际上步长为100秒,每帧代表1000秒,按60帧计算,则仿真时间1秒等于实际时间$6\times10^4$秒,地球速度按30km/s算则仿真时间1秒内地球走过$1.8\times10^6$km,换算回仿真中则走过1.8求解器距离。
换一种方法计算,地球的求解器速度是$v_1=10^{-2}\times30=0.3$,仿真软件每1秒仿真 :60帧 $\times$ 10步/帧 $\times$ 0.01单位时间/步=6个时间单位,也可以得到仿真时间1秒内地球走过1.8求解器距离。

%%%%%%%%%%%%%%%%%%%%%%%%%%%%%%%%%%%%%%%%%%%%%%%%%%%%%%%%%%%%%%%%
% 节
%%%%%%%%%%%%%%%%%%%%%%%%%%%%%%%%%%%%%%%%%%%%%%%%%%%%%%%%%%%%%%%%
\section{轨道六根数和位置速度向量}
轨道六根数指:半长轴(a)、偏心率(e),轨道倾角(i),升交点赤经($\Omega$)、近地点幅角($omega$)、真近点角($\phi$)。

%%%%%%%%%%%%%%%%%%%%%%%%%%%%%%%%%%%%%%%%%%%%%%%%%%%%%%%%%%%%%%%%
% 节
%%%%%%%%%%%%%%%%%%%%%%%%%%%%%%%%%%%%%%%%%%%%%%%%%%%%%%%%%%%%%%%%
\section{结\quad 论}
本文。

%%%%%%%%%%%%%%%%%%%%%%%%%%%%%%%%%%%%%%%%%%%%%%%%%%%%%%%%%%%%%%%%
% 参考文献
%%%%%%%%%%%%%%%%%%%%%%%%%%%%%%%%%%%%%%%%%%%%%%%%%%%%%%%%%%%%%%%%
\bibliographystyle{stylebib}
\bibliography{reference}
