\section{基本公式推导}
下面推导常用的一些公式。其中
角速度公式和相对加速度公式用于飞行器再入动力学方程,
递推最小二乘公式用于基于特征模型的全系数自适应控制。

% ////////////////////////////////////////
\subsection{角速度公式}
设一向量$\vec{x}(t)$绕旋转轴$\vec{\omega}$作匀速圆周运动,
则$\vec{x}(t)$的线速度为
\[\dot{\vec{x}}(t)=\vec{\omega}\times\vec{x}(t)\]
\textbf{证明}:由罗德里格斯((Rodrigues)旋转公式
\[R=\cos\theta I+(1-\cos\theta)\vec{n}\vec{n}^\text{T}+\sin\theta\vec{n}^{\wedge}\]
其中
$R$为旋转矩阵,
$n$为单位旋转轴,
$\theta$为旋转角度,
$\vec{n}^{\wedge}$表示向量$\vec{n}$叉乘对应的反对称矩阵。
将向量$\vec{\omega}$写成$\vec{\omega}=\omega\vec{n}$,
并将旋转矩阵对时间$t$求导得到
\begin{align*}
    \frac{\text{d}}{\text{d}t}R(t)
    =& \frac{\text{d}}{\text{d}t}\left(
        \cos\omega t I
        +(1-\cos\omega t)\frac{\vec{\omega}}{\omega}\frac{\vec{\omega}^\text{T}}{\omega}
        +\sin\omega t\frac{\vec{\omega}^{\wedge}}{\omega}
    \right) \\
    =& -\omega\sin\omega tI
        +\frac{\sin\omega t}{\omega}\vec{\omega}\vec{\omega}^\text{T}
        +\cos\omega t\cdot\vec{\omega}^{\wedge} \\
    \dot{R}(t)\vec{x}_0
    =& -\omega\sin\omega t\vec{x}_0
        +\frac{\sin\omega t}{\omega}\vec{\omega}^\text{T}\vec{x}_0\vec{\omega}
        +\cos\omega t\cdot\vec{\omega}\times\vec{x}_0
\end{align*}
另因为
\begin{align*}
    \vec{\omega}\times R(t)\vec{x}_0
    =& \cos\omega t\cdot\vec{\omega}\times\vec{x}_0
        +\frac{1-\cos\omega t}{\omega^2}\vec{\omega}\times\vec{\omega}\vec{\omega}^\text{T}\vec{x}_0 \\
        &+ \frac{\sin\omega t}{\omega}\vec{\omega}\times(\vec{\omega}\times\vec{x}_0) \\
    =& \cos\omega t\cdot\vec{\omega}\times\vec{x}_0
        +\frac{\sin\omega t}{\omega}(\vec{\omega}^\text{T}\vec{x}_0\vec{\omega}
        -\vec{\omega}^\text{T}\vec{\omega}\vec{x}_0)
\end{align*}
所以
\[\dot{\vec{x}}(t)=\dot{R}(t)\vec{x}_0=\vec{\omega}\times R(t)\vec{x}_0=\vec{\omega}\times\vec{x}(t)\]

% ////////////////////////////////////////
\subsection{旋转坐标系下的速度和加速度}
旋转坐标系$\mathcal{F}_2$绕惯性坐标系$\mathcal{F}_1$以角速度$\vec{\omega}$旋转,
设$\mathcal{F}_1$下位置向量的一阶导和二阶导分别为
$\dot{\vec{r}}$和$\ddot{\vec{r}}$,
$\mathcal{F}_2$下位置向量的一阶导和二阶导为
$\overset{\circ}{\vec{r}}$和$\overset{\circ\circ}{\vec{r}}$,
满足
\begin{align*}
    \dot{\vec{r}}
    =& \overset{\circ}{\vec{r}}
    + \vec{\omega}\times\vec{r} \\
    \ddot{\vec{r}}
    =& \overset{\circ\circ}{\vec{r}}
    + 2\vec{\omega}\times\overset{\circ}{\vec{r}}
    + \overset{\circ}{\vec{\omega}}\times\vec{r}
    + \vec{\omega}\times(\vec{\omega}\times\vec{r})
\end{align*}
\textbf{证明}:
设同一向量$\vec{r}$在坐标系$\mathcal{F}_1$和$\mathcal{F}_2$下的坐标分别为
\begin{equation*}
    \vec{r} = \left[\begin{matrix}
        \vec{e}_x & \vec{e}_y & \vec{e}_z
    \end{matrix}\right]
    \left[\begin{matrix}
        x_1 \\ y_1 \\ z_1
    \end{matrix}\right]
    = \left[\begin{matrix}
        \vec{a}_x & \vec{a}_y & \vec{a}_z
    \end{matrix}\right]
    \left[\begin{matrix}
        x_2 \\ y_2 \\ z_2
    \end{matrix}\right]
\end{equation*}
其中$[\vec{e}_x\ \vec{e}_y\ \vec{e}_z]$表示惯性坐标系$\mathcal{F}_1$下的三轴单位向量,
$[\vec{a}_x\ \vec{a}_y\ \vec{a}_z]$表示坐标系$\mathcal{F}_2$的三轴单位向量在惯性坐标系$\mathcal{F}_1$下的坐标,
三个单位向量张成旋转坐标系$\mathcal{F}_2$,
则向量$\vec{r}$的一阶导
\begin{align*}
    \dot{\vec{r}}
    =& \left[\begin{matrix}
        \dot{\vec{a}}_x & \dot{\vec{a}}_y & \dot{\vec{a}}_z
    \end{matrix}\right]
    \left[\begin{matrix}
        x_2 \\ y_2 \\ z_2
    \end{matrix}\right]
    + \left[\begin{matrix}
        \vec{a}_x & \vec{a}_y & \vec{a}_z
    \end{matrix}\right]
    \left[\begin{matrix}
        \dot{x}_2 \\ \dot{y}_2 \\ \dot{z}_2
    \end{matrix}\right] \\
    =& \vec{\omega}\times
    \left[\begin{matrix}
        \vec{a}_x & \vec{a}_y & \vec{a}_z
    \end{matrix}\right]
    \left[\begin{matrix}
        x_2 \\ y_2 \\ z_2
    \end{matrix}\right]
    + \overset{\circ}{\vec{r}} \\
    =& \vec{\omega}\times\vec{r} + \overset{\circ}{\vec{r}}
\end{align*}
使用坐标系的记法写作
\begin{align*}
    \dot{\vec{r}}
    =& \frac{\text{d}}{\text{d}t}(\mathcal{F}_2\vec{r}_2) \\
    =& \mathcal{F}_2\dot{\vec{r}}_2
    + \dot{\mathcal{F}_2}\vec{r}_2 \\
    =& \mathcal{F}_2\dot{\vec{r}}_2
    + \vec{\omega} \times \mathcal{F}_2\vec{r}_2 \\
    =& \mathcal{F}_2\dot{\vec{r}}_2
    + \vec{\omega} \times \mathcal{F}_1\vec{r}_1
\end{align*}
对上式进一步求导得
\begin{align*}
    \ddot{\vec{r}}
    =& \vec{\omega} \times \mathcal{F}_2\dot{\vec{r}}_2
    + \mathcal{F}_2\ddot{\vec{r}}_2
    + \dot{\vec{\omega}} \times \mathcal{F}_1\vec{r}_1 \\
    &+ \vec{\omega} \times (\vec{\omega} \times \mathcal{F}_2\vec{r}_2
    + \mathcal{F}_2\dot{\vec{r}}_2) \\
    =& \mathcal{F}_2\ddot{\vec{r}}_2
    + 2\vec{\omega} \times \mathcal{F}_2\dot{\vec{r}}_2 \\
    &+ \dot{\vec{\omega}} \times \mathcal{F}_1\vec{r}_1
    + \vec{\omega} \times (\vec{\omega} \times \mathcal{F}_1\vec{r}_1) \\
    =& \overset{\circ\circ}{\vec{r}}
    + 2\vec{\omega}\times\overset{\circ}{\vec{r}}
    + \dot{\vec{\omega}}\times\vec{r}
    + \vec{\omega}\times(\vec{\omega}\times\vec{r})
\end{align*}
其中$\vec{\omega}$在两个坐标系下的坐标相等,
即$\dot{\vec{\omega}}=\overset{\circ}{\vec{\omega}}$。

% ////////////////////////////////////////////////////////////////
\subsection{轨道六根数和位置速度向量} \label{secFormElement}
轨道六根数指:
半长轴(a)、偏心率(e),轨道倾角(i),升交点赤经($\Omega$)、近地点幅角($\omega$)、真近点角($\phi$)。
根据文献\cite{mruiter2012}中的公式计算相关参数。
由下式计算轨道能量
\begin{equation*}
    \mathcal{E}=-\frac{\mu}{2a}=\frac{v_0^2}{2}-\frac{\mu}{r_0} = \frac{v_{inf}^2}{2}
\end{equation*}
由下式计算角动量绝对值
\begin{align*}
    \vec{h} =& \vec{r}\times\vec{v} \\
    |\vec{h}| =& \vec{r}\cdot\vec{v}\cos\gamma
\end{align*}
其中,$h$为角动量,$\gamma$为进入点航迹角,
$\vec{r}$和$\vec{v}$分别为进入点位置和速度向量。
计算双曲线轨道偏心率$e$
\begin{align*}
    &a=\frac{p}{1-e^2},p=\frac{h^2}{\mu} \\
    &e=\sqrt{1-\frac{h^2}{a\mu}}
\end{align*}
其中$p$称为半通径。
计算真近点角$f$
\begin{align*}
    &r=\frac{p}{1+e\cos f} \\
    &f=\arccos(\frac{1}{e}(\frac{h^2}{\mu r}-1))
\end{align*}
由轨道元素计算位置和速度向量
\begin{align}
    R =& \left[\begin{matrix}
        c_\Omega c_\omega-s_\Omega c_i s_\omega & -c_\Omega s_\omega-s_\Omega c_i c_\omega & s_\Omega s_i \\
        s_\Omega c_\omega+c_\Omega c_i s_\omega & -s_\Omega s_\omega+c_\Omega c_i c_\omega & -c_\Omega s_i \\
        s_i s_\omega & s_i c_\omega & c_i
    \end{matrix}\right] \notag\\
    \vec{r} =& R\left[\begin{matrix}
        \frac{a(1-e^2)}{1+e\cos\theta}\cos\theta \\ \frac{a(1-e^2)}{1+e\cos\theta}\sin\theta \\ 0
    \end{matrix}\right] \notag\\
    \vec{v} =& R\left[\begin{matrix}
        -\sqrt{\frac{\mu}{a(1-e^2)}}\sin\theta \\ \sqrt{\frac{\mu}{a(1-e^2)}}(e+\cos\theta) \\ 0
    \end{matrix}\right] \label{eqFormEle2RV}
\end{align}
根据平近点角计算真近点角\cite{msmart1977}
\begin{align*}
    f =& M+\left(2e-{\frac {1}{4}}e^{3}\right)\sin {M}
    + {\frac {5}{4}}e^{2}\sin {2M} \\
    &+ {\frac {13}{12}}e^{3}\sin {3M}+O(e^{4})
\end{align*}

% ////////////////////////////////////////////////////////////////
\subsection{递推最小二乘公式}
对多组数据$\boldsymbol{x}$和$y$,满足
\[y_i = \boldsymbol{x}^\mathrm{T}_i\boldsymbol{\theta}\]
其中$\boldsymbol{x}$是输入数据向量,$y$是输出数据标量。写成矩阵形式
\[\boldsymbol{y} = \mathbf{X}\theta\]
其中
\begin{align*}
    \mathbf{X} =& \left[\begin{matrix}
        \boldsymbol{x}^\mathrm{T}_1 \\
        \boldsymbol{x}^\mathrm{T}_2 \\
        \boldsymbol{x}^\mathrm{T}_3
    \end{matrix}\right] \\
    \mathbf{X}^\mathrm{T}_k =& \left[\begin{matrix}
        \boldsymbol{x}_1 & \boldsymbol{x}_2 & \boldsymbol{x}_3 &
        \cdots & \boldsymbol{x}_k
    \end{matrix}\right]
\end{align*}
最小二乘解为
\begin{equation*}
    \boldsymbol{\theta}
    =(\mathbf{X}^\mathrm{T}\mathbf{X})^{-1}
    \mathbf{X}^\mathrm{T}\boldsymbol{y}\tag{2.1}
\end{equation*}
令
\begin{align*}
    \mathbf{P}^{-1} =& \mathbf{X}^\mathrm{T}\mathbf{X} \\
    \mathbf{P}_k^{-1} =& \sum_{i=1}^k\boldsymbol{x}_i\boldsymbol{x}^\mathrm{T}_i
    = \sum_{i=1}^{k-1}\boldsymbol{x}_i\boldsymbol{x}^\mathrm{T}_i
    +\boldsymbol{x}_k\boldsymbol{x}^\mathrm{T}_k \\
    =& \mathbf{P}_{k-1}^{-1} + \boldsymbol{x}_k\boldsymbol{x}_k^\mathrm{T}
\end{align*}
同理可得
\begin{equation*}
    \mathbf{X}_k^\mathrm{T}\boldsymbol{y}_k
    =\mathbf{X}_{k-1}^\mathrm{T}\boldsymbol{y}_{k-1}
    +\boldsymbol{x}_ky_k
\end{equation*}
于是代入式(1)得到
\begin{align*}
\boldsymbol{\theta}_k =& \mathbf{P}_k\mathbf{X}_k^\mathrm{T}\boldsymbol{y}_k \\
=& \mathbf{P}_k(\mathbf{X}_{k-1}^\mathrm{T}\boldsymbol{y}_{k-1}
 +\boldsymbol{x}_ky_k) \\
=& \mathbf{P}_k(\mathbf{P}_{k-1}^{-1}\boldsymbol{\theta}_{k-1}
 +\boldsymbol{x}_ky_k) \\
=& \mathbf{P}_k(\mathbf{P}_k^{-1}\boldsymbol{\theta}_{k-1}
 -\boldsymbol{x}_k\boldsymbol{x}_k^\mathrm{T}\theta_{k-1}+\boldsymbol{x}_ky_k) \\
=& \boldsymbol{\theta}_{k-1} + \mathbf{P}_k\boldsymbol{x}_k
 (y_k-\boldsymbol{x}_k^\mathrm{T}\theta_{k-1}) \\
=& \boldsymbol{\theta}_{k-1}
 +(\mathbf{P}_{k-1}^{-1} + \boldsymbol{x}_k\boldsymbol{x}_k^\mathrm{T})^{-1}
 \boldsymbol{x}_k(y_k-\boldsymbol{x}_k^\mathrm{T}\theta_{k-1}) \\
=& \boldsymbol{\theta}_{k-1}
 +(\mathbf{P}_{k-1}-\frac{\mathbf{P}_{k-1}\boldsymbol{x}_k\boldsymbol{x}_k^\mathrm{T}
 \mathbf{P}_{k-1}}{1+\boldsymbol{x}_k^\mathrm{T}\mathbf{P}_{k-1}\boldsymbol{x}_k})
 \boldsymbol{x}_k(y_k-\boldsymbol{x}_k^\mathrm{T}\theta_{k-1}) \\
\end{align*}
其中
\begin{equation*}
    \mathbf{P}_k =
 \mathbf{P}_{k-1}-\frac{\mathbf{P}_{k-1}\boldsymbol{x}_k\boldsymbol{x}_k^\mathrm{T}
 \mathbf{P}_{k-1}}{1+\boldsymbol{x}_k^\mathrm{T}\mathbf{P}_{k-1}\boldsymbol{x}_k}
\end{equation*}
用到了下面的矩阵求逆公式及其引理
\begin{align*}
(\mathbf{A}+\mathbf{B C D})^{-1} =& \mathbf{A}^{-1}-\mathbf{A}^{-1} \mathbf{B}
(\mathbf{D A} \mathbf{A}^{-1} \mathbf{B}+\mathbf{C}^{-1})^{-1} \mathbf{D A}^{-1} \\
(\mathbf{A}+\mathbf{u} \mathbf{u}^{T})^{-1} =& \mathbf{A}^{-1}
 -\frac{\mathbf{A}^{-1} \mathbf{u} \mathbf{u}^{T} \mathbf{A}^{-1}}
 {1+\mathbf{u}^{T} \mathbf{A}^{-1} \mathbf{u}}
\end{align*}
令
\begin{equation*}
    \mathbf{K}_k = \frac{\mathbf{P}_{k-1}\boldsymbol{x}_k}
{1+\boldsymbol{x}_k^\mathrm{T}\mathbf{P}_{k-1}\boldsymbol{x}_k}
\end{equation*}
则
\begin{align*}
\mathbf{P}_k =& (\mathbf{I}-\mathbf{K}_k\boldsymbol{x}_k^\mathrm{T})\mathbf{P}_{k-1}\\
\mathbf{P}_k\boldsymbol{x}_k =& \mathbf{P}_{k-1}\boldsymbol{x}_k
 -\mathbf{K}_k\boldsymbol{x}_k^\mathrm{T}\mathbf{P}_{k-1}\boldsymbol{x}_k \\
=& \frac{\mathbf{P}_{k-1}\boldsymbol{x}_k
 (1+\boldsymbol{x}_k^\mathrm{T}\mathbf{P}_{k-1}\boldsymbol{x}_k)
 -\mathbf{P}_{k-1}\boldsymbol{x}_k\boldsymbol{x}_k^\mathrm{T}
 \mathbf{P}_{k-1}\boldsymbol{x}_k}
 {1+\boldsymbol{x}_k^\mathrm{T}\mathbf{P}_{k-1}\boldsymbol{x}_k} \\
=& \frac{\mathbf{P}_{k-1}\boldsymbol{x}_k}
 {1+\boldsymbol{x}_k^\mathrm{T}\mathbf{P}_{k-1}\boldsymbol{x}_k} \\
=& \mathbf{K}_k \\
\end{align*}
总结成递推公式得到
\begin{align*}
\mathbf{K}_k =& \frac{\mathbf{P}_{k-1}\boldsymbol{x}_k}{1+\boldsymbol{x}_k^\mathrm{T}
\mathbf{P}_{k-1}\boldsymbol{x}_k} \\
\mathbf{P}_k =& (\mathbf{I}-\mathbf{K}_k\boldsymbol{x}_k^\mathrm{T})\mathbf{P}_{k-1}\\
\boldsymbol{\theta}_k =& \boldsymbol{\theta}_{k-1}
 +\mathbf{K}_k(y_k-\boldsymbol{x}_k^\mathrm{T}\theta_{k-1})
\end{align*}

