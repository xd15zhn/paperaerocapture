%%%%%%%%%%%%%%%%%%%%%%%%%%%%%%%%%%%%%%%%%%%%%%%%%%%%%%%%%%%%%%%%
% Sim
%%%%%%%%%%%%%%%%%%%%%%%%%%%%%%%%%%%%%%%%%%%%%%%%%%%%%%%%%%%%%%%%
\section{仿真}
本文使用微分方程求解器Simucpp\cite{olzhn2021}进行仿真。
下文介绍各部分设计原理。

% ////////////////////////////////////////
\subsection{被控对象建模}
可以用第2节中推导出的式\eqref{eqModelTotal}建立被控对象微分方程,
若忽略火星自转的影响,也可以建立向量形式的微分方程
\begin{equation}
    \ddot{\vec{r}} = \frac{\mu}{||\vec{r}||^3}\vec{r}+\vec{L}+\vec{D} \label{eqSimFA}
\end{equation}
其中阻力向量为
\begin{align}
    \vec{D} = -D\frac{\vec{v}}{||\vec{v}||} \label{eqSimFD}
\end{align}
升力向量$\vec{L}$与铅垂面夹角为倾侧角$\sigma$,
为了能将$\vec{L}$用其他已知向量表示,
需要构造一对与$\vec{L}$同平面的正交单位向量。
已知$\vec{v}$与$\vec{r}$张成铅垂面,
另构造一个垂直于铅垂面且在惯性系中方向向上的单位向量$\vec{n}_2$,
和另一个与$\vec{r}$同方向的单位向量$\vec{n}_1$,
则$\vec{n}_1$位于铅垂面内,
此时$\vec{L}$与$\vec{n}_1$的夹角即为倾侧角,
$\vec{n}_1$和$\vec{n}_2$即为用于表示$\vec{L}$的正交单位向量,
即
\begin{align}
    \vec{n}_2 =& \frac{\vec{r}\times\vec{v}}{||\vec{r}\times\vec{v}||} \notag\\
    \vec{L} =& \vec{n}_1\cos\sigma + \vec{n}_2\sin\sigma \label{eqSimFL}
\end{align}
微分方程式\eqref{eqSimFA}\eqref{eqSimFD}\eqref{eqSimFL}共同组成被控对象模型。
建立飞行器被控对象模型的模块框图如图\ref{figSimPlant}所示。
\begin{center}
	\includegraphics[scale=0.8]{plant.pdf}  \\
	\figcaption{被控对象模型的模块框图}\label{figSimPlant}
\end{center}
图中,被控对象的输入(控制量)为倾侧角$\sigma$,
输出为位置$\vec{r}$。
图中的两个多输入单输出函数分别为式\eqref{eqSimFLDTotal}\eqref{eqSimFATotal}
\begin{align}
    \left\{\begin{aligned}
    h =& ||\vec{r}||-R \\
    \rho =& \rho_0e^{-h/h_s} \\
    D =& \frac{1}{2}\rho||\vec{v}||^2S_{\text{ref}}C_D \\
    \vec{D} =& f_1(\vec{r},\vec{v}) = -D\frac{\vec{v}}{||\vec{v}||} \\
    L =& DC_{LD} \\
    \vec{n}_1 =& \frac{\vec{r}}{||\vec{r}||} \\
    \vec{n}_2 =& \frac{\vec{r}\times\vec{v}}{||\vec{r}\times\vec{v}||} \\
    \vec{L} =& f_2(\vec{r},\vec{v},\sigma) = L(\vec{n}_1\cos\sigma + \vec{n}_2\sin\sigma) \\
    \vec{f}_{LD} =& f_{LD}(\vec{r},\vec{v},\sigma) = \vec{L} + \vec{D}
\end{aligned}\right. \label{eqSimFLDTotal}
\end{align}
\begin{equation}
    \ddot{\vec{r}} = f_A(\vec{r},\vec{f}_{LD}) = \frac{\mu}{||\vec{r}||^3}\vec{r}+\frac{\vec{f}_{LD}}{m} \label{eqSimFATotal}
\end{equation}
式中$R$为火星半径,$C_{LD}=C_L/C_D$为升阻比。

% ////////////////////////////////////////
\subsection{简化模型仿真结果}
给定进入点轨道初值和控制目标如下。
\begin{center}\begin{tabular}{lll}
    \toprule
    条件 & 初值 & 目标 \\
    \midrule
    进入点速度(km/s) & 5.8 \\
    进入点高度(km) & 125 \\
    进入航迹角$\gamma$(rad) & 0.192(11$^\circ$) \\
    偏心率 & 1.75 & 0.215 \\
    轨道倾角$i$(rad) & 0.524(30$^\circ$) & 0.524 \\
    升交点赤经(rad) & 0 & \\
    近拱点幅角(rad) & 0 & \\
    远拱点高度(km) & & 2250 \\
    近拱点高度(km) & & 250 \\
    \bottomrule
\end{tabular}\end{center}
设定参数中,偏心率可由进入点速度、进入点高度、进入航迹角得出,
由第\ref{secFormElement}小节公式可验证设定参数基本正确。
本文仿真暂不考虑轨道倾角。

首先展示不使用制导律,控制量输出常值的仿真结果,取$\sigma=1$。
3D仿真结果如图\ref{figSim1View}所示。
\begin{center}
	\includegraphics[scale=0.2]{sim1Verticalview.png}  \\
	\includegraphics[scale=0.2]{sim1Sideview.png}  \\
	\figcaption{确定轨迹3D仿真结果}\label{figSim1View}
\end{center}
图\ref{figSim1View}中的两幅图分别为正等轴测图和侧视图。
轨道高度变化曲线如图\ref{figSim1Height}所示。
横坐标为时间,单位为0.1秒,截取了前60秒。
\begin{center}
	\includegraphics[scale=0.6]{sim1Height.pdf} \\
	\figcaption{轨道高度变化曲线}\label{figSim1Height}
\end{center}

% ////////////////////////////////////////
\subsection{制导律仿真结果}
制导律与各种仿真参数见程序。
期望远拱点半径与制导律作用下的实际远拱点半径均为$5652.5$km。
图\ref{figSim2State}所示分别为高度、倾侧角、侧向速度的变化曲线,
原始数据的单位分别为km、rad、$10^{-4}$km/s。
此时未加侧向制导的约束。
\begin{center}
	\includegraphics[scale=0.6]{Sim2State.pdf} \\
	\figcaption{未加侧向制导约束时的多种状态曲线}\label{figSim2State}
\end{center}
加入侧向制导约束后的结果如图\ref{figSim2State2}所示。
\begin{center}
	\includegraphics[scale=0.6]{Sim2State2.pdf} \\
	\figcaption{加入侧向制导约束后的多种状态曲线}\label{figSim2State2}
\end{center}
3D仿真结果如图\ref{figSim2View}所示,两幅图分别为俯视图和侧视图。
\begin{center}
	\includegraphics[scale=0.2]{sim2Verticalview.png}  \\
	\includegraphics[scale=0.2]{sim2Sideview.png}  \\
	\figcaption{制导律3D仿真结果}\label{figSim2View}
\end{center}
对精确模型的飞行器飞出大气层后的完整轨迹,使用解析法计算过程较复杂,
所以大气层内外全程采用数值解,计算量较大,
而且数值解难以计算飞行器在什么时候飞到了远拱点。
数值解步长为$10^{-3}$,一个步长代表$0.01$秒,
大气层内每100个步长、大气层外每1000个步长记录一次坐标。
从飞出大气层时开始计算的4000秒时飞行器的向径为$5650.39$km,
也就是说图\ref{figSim2View}中的轨迹末端并不是严格的远拱点。

以后还将加入大气变化等不确定性因素,
时间有限暂不展开。
