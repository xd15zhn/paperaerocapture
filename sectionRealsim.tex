%%%%%%%%%%%%%%%%%%%%%%%%%%%%%%%%%%%%%%%%%%%%%%%%%%%%%%%%%%%%%%%%
% Real
%%%%%%%%%%%%%%%%%%%%%%%%%%%%%%%%%%%%%%%%%%%%%%%%%%%%%%%%%%%%%%%%
\section{实时仿真理论}
宇宙中各天体的自转或公转周期通常以天甚至以年计算,
且轨道速度快,空间尺度大。
如果按照实际参数进行仿真,
则仿真的计算量太大而无法实时计算。
为了能够在仿真时直观地实时展示仿真结果,
需要对一些公式的单位进行换算。

万有引力公式
\begin{equation*}
    \frac{\text{d}^2\vec{r}}{\text{d}t^2}=-\frac{\mu}{r^3}\vec{r}
\end{equation*}
中,$r$的单位是km,$t$的单位是s。
因为实际的尺度太大,一个天文单位达到了$10^8$数量级,
为了便于仿真,
需要将实际尺度中的时间和距离变换到一个合适的时空坐标系下。
首先列举一些名词解释。
\textbf{实际时间}指宇宙尺度上的实际时间;
\textbf{无单位求解器时间}指微分方程求解器中自变量的值,
没有单位,简称求解器时间,
对应的求解器速度和求解器距离也都没有单位;
\textbf{仿真展示软件消耗时间}:例如仿真展示软件运行1秒对应实际时间的1000秒,
此处的1秒指的是是仿真展示软件消耗时间,简称展示时间。

举例说明,
求解器时间的1单位时间等于实际时间的$10^4$秒,
1单位距离等于实际距离$10^6$km,即在求解器中,
\begin{equation}
    \bar{t}=10^{-4}t,\ \bar{r}=10^{-6}r \label{eqRealConvert}
\end{equation}
其中$t$和$r$分别为以秒和千米为单位的实际时间和实际距离,
而$\bar{t}$和$\bar{r}$分别表示求解器时间和求解器距离。
1个天文单位为$r=1.5\times10^8$km,
则求解器中的1个天文单位为$\bar{r}=150$。
然后可以计算出实际时间/速度/距离
和求解器时间/速度/距离之间的一些换算关系
\begin{align*}
    &\bar{v} = \frac{\text{d}\bar{r}}{\text{d}\bar{t}}
     = \frac{10^{-6}\text{d}r}{10^{-4}\text{d}t} = 10^{-2}v \\
    &\bar{a} = \frac{\text{d}}{\text{d}\bar{t}}\frac{\text{d}\bar{r}}{\text{d}\bar{t}}
     = 10^2\frac{\text{d}^2r}{\text{d}t^2} = 10^2a \\
    &\bar{\mu}\left[\frac{(10^{-6}\text{km})^3}{(10^{-4}\text{s})^2}\right]
     = 10^{-10}\mu\left[\frac{\text{km}^3}{\text{s}^2}\right] \\
    &-\frac{\bar{\mu}}{\bar{r}^2} = -10^2\frac{\mu}{r^2}
     = 10^2\frac{\text{d}^2r}{\text{d}t^2} = \bar{a}
    \end{align*}
    其中引力常数$\mu$的单位中包含时间/距离量纲,
因此也需要换算。
可以验证,实际中的微分方程
$$a=-\frac{\mu}{r^2}$$
变换到求解器中仍然为
$$\bar{a}=-\frac{\bar{\mu}}{\bar{r}^2}$$

下面在换算公式\eqref{eqRealConvert}的基础上引入仿真展示软件消耗时间。
若仿真步长设置为0.01求解器时间,仿真展示软件每帧仿真10步,
则换算到实际时间上的步长为100秒,每帧代表1000秒,
按60帧计算,则展示时间1秒等于实际时间$6\times10^4$秒,
地球的轨道速度按30km/s算则展示时间1秒内地球走过$1.8\times10^6$km,
换算回仿真中则走过1.8展示距离。
换一种方法计算,
地球的求解器速度是$\bar{v}=10^{-2}\times30=0.3$,
仿真展示软件每1秒仿真:
60帧$\times$10步/帧$\times$0.01单位时间/步=6个时间单位,
也可以得到展示时间1秒内地球走过1.8展示距离。
在式\eqref{eqRealConvert}的设定下,
仿真展示软件每帧仿真1步与10步对应的地球自转周期分别约为$14.4$和$1.44$秒,
分别可用于展示低轨卫星和地球同步卫星的动态运行结果。

与实时仿真不同的是实际中广泛使用的非实时仿真,
也就是在一次仿真结束后展示静态仿真结果。
此时虽然不需要考虑展示时间/速度/距离,
但求解器时间/速度/距离仍然需要考虑。
如果仿真需要展示系统运动的动态过程,
则应使用实时仿真,否则应使用非实时仿真。
本文中火星气动捕获制导结果仅需要展示最终制导轨迹,
因此使用使用非实时仿真,
不需要考虑展示时间/速度/距离。

本文火星气动捕获的仿真中需要考虑两种微分方程模型。
第一个是简化模型,仿真时间较短,
倾侧角输出常数,然后计算远拱点误差并应用于制导律;
第二个是精确模型,仿真时间较长,
将预测校正制导应用于精确模型中并展示仿真结果。
两个模型中时间分别加快$10^4$倍和10倍,
为了使两个模型尽可能保持一致从而方便编程,
保持两者的距离单位均为实际距离。
第二个模型中,离散制导律的周期为1秒,
连续系统的求解步长取离散系统的$1/100$,
仿真器的求解器步长取$10^{-3}$,则离散系统周期为$0.1$。

简化模型中,式\eqref{eqRealConvert}中取
$\bar{r}=r$,$\bar{t}=10^{-4}t$,
对应的一些变量和常数的换算关系如下
\begin{align*}
    &\bar{v} = 10^4v \\
    &\bar{a} = 10^8a \\
    &\bar{\mu}\left[\frac{(\text{km})^3}{(10^{-4}\text{s})^2}\right]
     = 10^8\mu\left[\frac{\text{km}^3}{\text{s}^2}\right] \\
    &-\frac{\bar{\mu}}{\bar{r}^2} = -10^8\frac{\mu}{r^2} = 10^8a = \bar{a} \\
\end{align*}

精确模型中,式\eqref{eqRealConvert}中取
$\bar{r}=r$,$\bar{t}=0.1t$,
对应的一些变量和常数的换算关系如下
\begin{align*}
    &\bar{v} = 10v \\
    &\bar{a} = 10^2a \\
    &\bar{\mu}\left[\frac{(\text{km})^3}{(0.1\text{s})^2}\right]
     = 100\mu\left[\frac{\text{km}^3}{\text{s}^2}\right] \\
    &-\frac{\bar{\mu}}{\bar{r}^2} = -100\frac{\mu}{r^2} = 100a = \bar{a} \\
\end{align*}

另由量纲运算可得
升力系数、阻力系数、升阻比均无单位,不需要换算。

单位换算在显著加快仿真速度的同时,
微分方程数值求解的精度也会受到影响。
已知四阶龙格库塔法的局部截断误差为$O(h^5$)\cite{mqingyang2019},
取无量纲步长为$h_0=10^{-3}$,
对微分方程
\[\frac{\text{d}\bar{r}}{\text{d}\bar{t}}=\bar{v}\]
局部截断误差为$\Delta\bar{r}=Ch_0^5=10^{-15}C$,
换算回真实值得到$\Delta r=10^6\Delta\bar{r}=10^{-9}C$,
同理可得
$\Delta v=10^2\Delta\bar{v}=10^{-13}C$,
$\Delta a=10^{-2}\Delta\bar{a}=10^{-17}C$。
以$r$为例,
换算单位前仿真10秒的累积误差为
\[e_1=10^4C\times 10^{-15}=10^{-11}C\]
换算单位后,时间单位换算成万秒,
仍取无量纲步长$h_0=10^{-3}$,
即实际步长为$h_2=10^{-3}$万秒,
仿真10秒就是1个步长,累积误差等于局部截断误差
\[e_2=10^{-9}C\]
若不换算单位而只是改变步长,
取步长$h_3=1$秒,
尽管实际上$h_3=10h_2$,
但仿真10秒的累积误差高达
\[e_3=10C\times 1^5=10C\]
因此在不同的仿真中,
在数值求解的仿真步长不变的前提下,
使用单位换算的方法可以显著提高仿真速度,
同时也能保证仿真精度不会显著下降,
对于一些高阶导量甚至提高了仿真精度。
(对这一现象我不是很理解)

原文\cite{dqingyuan2019}中将被控对象的仿真步长设为1,
我不赞同原文的做法,认为步长为1的误差很大。
