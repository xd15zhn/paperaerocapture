\section{飞行器再入动力学方程}
本节推导不考虑火星自转时的飞行器再入动力学方程,
假设飞行器只受空气动力和重力,无风干扰。

首先建立坐标系。

1. 火星赤道惯性坐标系($O-X_IY_IZ_I$):
原点$O$位于火星球心,
$OX_I$轴指向火星赤道平面与黄道平面相交节线的升交点,
$OZ_I$轴垂直于赤道平面指向北极,
$OY_I$轴与$OX_I$和$OZ_I$轴组成右手正交坐标系。
该坐标系与地球赤道惯性坐标系定义相仿。

2. 火星赤道固连坐标系($O-XYZ$):
原点$O$位于火星球心,
$OX$轴在赤道平面内指向零经度线,
$OZ$轴垂直于赤道平面指向北极,
$OY$轴与$OX$和$OZ$轴组成右手正交坐标系。
该坐标系与火星固连,
并相对火星赤道惯性坐标系以火星自转角速度$\omega$转动。

3. 飞行器位置坐标系($O-xyz$):
又称"天东北坐标系",
原点$O$位于火星球心,
$Ox$轴指向飞行器质心$M$,
$Oy$轴平行于飞行器当地水平面的东向,
$Oz$轴平行于飞行器当地水平面的北向,

4. 速度坐标系($M-x_vy_vz_v$):
原点$M$位于飞行器质心,
$Mx_v$轴平行于$Ox$轴,
$Mz_v$轴指向飞行器速度方向,
$My_v$轴与$Mx_v$和$Mz_v$轴组成右手正交坐标系。

飞行器在三维空间中的运动可以由
位置$\boldsymbol{r}(t)$和
速度$\boldsymbol{v}(t)$两个向量来表征。
