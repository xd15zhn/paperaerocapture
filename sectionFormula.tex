\section{基本公式推导}
下面推导常用的一些公式。其中
角速度公式和相对加速度公式用于飞行器再入动力学方程,
递推最小二乘公式用于基于特征模型的全系数自适应控制\cite{mhongxin1990}。

% ////////////////////////////////////////
\subsection{角速度公式}
设一向量$\vec{x}(t)$绕旋转轴$\vec{\omega}$作匀速圆周运动,
则$\vec{x}(t)$的线速度为
\[\dot{\vec{x}}(t)=\vec{\omega}\times\vec{x}(t)\]
\textbf{证明}:由罗德里格斯((Rodrigues)旋转公式
\[R=\cos\theta I+(1-\cos\theta)\vec{n}\vec{n}^\text{T}+\sin\theta\vec{n}^{\wedge}\]
其中
$R$为旋转矩阵,
$n$为单位旋转轴,
$\theta$为旋转角度,
$\vec{n}^{\wedge}$表示向量$\vec{n}$叉乘对应的反对称矩阵。
将向量$\vec{\omega}$写成$\vec{\omega}=\omega\vec{n}$,
并将旋转矩阵对时间$t$求导得到
\begin{align*}
    \frac{\text{d}}{\text{d}t}R(t)
    =& \frac{\text{d}}{\text{d}t}\left(
        \cos\omega t I
        +(1-\cos\omega t)\frac{\vec{\omega}}{\omega}\frac{\vec{\omega}^\text{T}}{\omega}
        +\sin\omega t\frac{\vec{\omega}^{\wedge}}{\omega}
    \right) \\
    =& -\omega\sin\omega tI
        +\frac{\sin\omega t}{\omega}\vec{\omega}\vec{\omega}^\text{T}
        +\cos\omega t\cdot\vec{\omega}^{\wedge} \\
    \dot{R}(t)\vec{x}_0
    =& -\omega\sin\omega t\vec{x}_0
        +\frac{\sin\omega t}{\omega}\vec{\omega}^\text{T}\vec{x}_0\vec{\omega}
        +\cos\omega t\cdot\vec{\omega}\times\vec{x}_0
\end{align*}
另因为
\begin{align*}
    \vec{\omega}\times R(t)\vec{x}_0
    =& \cos\omega t\cdot\vec{\omega}\times\vec{x}_0
        +\frac{1-\cos\omega t}{\omega^2}\vec{\omega}\times\vec{\omega}\vec{\omega}^\text{T}\vec{x}_0 \\
        &+ \frac{\sin\omega t}{\omega}\vec{\omega}\times(\vec{\omega}\times\vec{x}_0) \\
    =& \cos\omega t\cdot\vec{\omega}\times\vec{x}_0
        +\frac{\sin\omega t}{\omega}(\vec{\omega}^\text{T}\vec{x}_0\vec{\omega}
        -\vec{\omega}^\text{T}\vec{\omega}\vec{x}_0)
\end{align*}
所以
\[\dot{\vec{x}}(t)=\dot{R}(t)\vec{x}_0=\vec{\omega}\times R(t)\vec{x}_0=\vec{\omega}\times\vec{x}(t)\]

% ////////////////////////////////////////
\subsection{旋转坐标系下的速度和加速度}
旋转坐标系$\mathcal{F}_2$绕惯性坐标系$\mathcal{F}_1$以角速度$\vec{\omega}$旋转,
设$\mathcal{F}_1$下位置向量的一阶导和二阶导分别为
$\dot{\vec{r}}$和$\ddot{\vec{r}}$,
$\mathcal{F}_2$下位置向量的一阶导和二阶导为
$\overset{\circ}{\vec{r}}$和$\overset{\circ\circ}{\vec{r}}$,
满足
\begin{align*}
    \dot{\vec{r}}
    =& \overset{\circ}{\vec{r}}
    + \vec{\omega}\times\vec{r} \\
    \ddot{\vec{r}}
    =& \overset{\circ\circ}{\vec{r}}
    + 2\vec{\omega}\times\overset{\circ}{\vec{r}}
    + \overset{\circ}{\vec{\omega}}\times\vec{r}
    + \vec{\omega}\times(\vec{\omega}\times\vec{r})
\end{align*}
\textbf{证明}:
设同一向量$\vec{r}$在坐标系$\mathcal{F}_1$和$\mathcal{F}_2$下的坐标分别为
\begin{equation*}
    \vec{r} = \left[\begin{matrix}
        \vec{e}_x & \vec{e}_y & \vec{e}_z
    \end{matrix}\right]
    \left[\begin{matrix}
        x_1 \\ y_1 \\ z_1
    \end{matrix}\right]
    = \left[\begin{matrix}
        \vec{a}_x & \vec{a}_y & \vec{a}_z
    \end{matrix}\right]
    \left[\begin{matrix}
        x_2 \\ y_2 \\ z_2
    \end{matrix}\right]
\end{equation*}
其中$[\vec{e}_x\ \vec{e}_y\ \vec{e}_z]$表示惯性坐标系$\mathcal{F}_1$下的三轴单位向量,
$[\vec{a}_x\ \vec{a}_y\ \vec{a}_z]$表示坐标系$\mathcal{F}_2$的三轴单位向量在惯性坐标系$\mathcal{F}_1$下的坐标,
三个单位向量张成旋转坐标系$\mathcal{F}_2$,
则向量$\vec{r}$的一阶导
\begin{align*}
    \dot{\vec{r}}
    =& \left[\begin{matrix}
        \dot{\vec{a}}_x & \dot{\vec{a}}_y & \dot{\vec{a}}_z
    \end{matrix}\right]
    \left[\begin{matrix}
        x_2 \\ y_2 \\ z_2
    \end{matrix}\right]
    + \left[\begin{matrix}
        \vec{a}_x & \vec{a}_y & \vec{a}_z
    \end{matrix}\right]
    \left[\begin{matrix}
        \dot{x}_2 \\ \dot{y}_2 \\ \dot{z}_2
    \end{matrix}\right] \\
    =& \vec{\omega}\times
    \left[\begin{matrix}
        \vec{a}_x & \vec{a}_y & \vec{a}_z
    \end{matrix}\right]
    \left[\begin{matrix}
        x_2 \\ y_2 \\ z_2
    \end{matrix}\right]
    + \overset{\circ}{\vec{r}} \\
    =& \vec{\omega}\times\vec{r} + \overset{\circ}{\vec{r}}
\end{align*}
使用坐标系的记法写作
\begin{align*}
    \dot{\vec{r}}
    =& \frac{\text{d}}{\text{d}t}(\mathcal{F}_2\vec{r}_2) \\
    =& \mathcal{F}_2\dot{\vec{r}}_2
    + \dot{\mathcal{F}_2}\vec{r}_2 \\
    =& \mathcal{F}_2\dot{\vec{r}}_2
    + \vec{\omega} \times \mathcal{F}_2\vec{r}_2 \\
    =& \mathcal{F}_2\dot{\vec{r}}_2
    + \vec{\omega} \times \mathcal{F}_1\vec{r}_1
\end{align*}
对上式进一步求导得
\begin{align*}
    \ddot{\vec{r}}
    =& \vec{\omega} \times \mathcal{F}_2\dot{\vec{r}}_2
    + \mathcal{F}_2\ddot{\vec{r}}_2
    + \dot{\vec{\omega}} \times \mathcal{F}_1\vec{r}_1 \\
    &+ \vec{\omega} \times (\vec{\omega} \times \mathcal{F}_2\vec{r}_2
    + \mathcal{F}_2\dot{\vec{r}}_2) \\
    =& \mathcal{F}_2\ddot{\vec{r}}_2
    + 2\vec{\omega} \times \mathcal{F}_2\dot{\vec{r}}_2 \\
    &+ \dot{\vec{\omega}} \times \mathcal{F}_1\vec{r}_1
    + \vec{\omega} \times (\vec{\omega} \times \mathcal{F}_1\vec{r}_1) \\
    =& \overset{\circ\circ}{\vec{r}}
    + 2\vec{\omega}\times\overset{\circ}{\vec{r}}
    + \dot{\vec{\omega}}\times\vec{r}
    + \vec{\omega}\times(\vec{\omega}\times\vec{r})
\end{align*}
其中$\vec{\omega}$在两个坐标系下的坐标相等,
即$\dot{\vec{\omega}}=\overset{\circ}{\vec{\omega}}$。

% ////////////////////////////////////////////////////////////////
\subsection{轨道六根数和位置速度向量} \label{secFormElement}
轨道六根数指:
半长轴(a)、偏心率(e),轨道倾角(i),升交点赤经($\Omega$)、近地点幅角($\omega$)、真近点角($\phi$)。
根据文献\cite{mruiter2012}中的公式计算相关参数。
由下式计算轨道能量
\begin{equation*}
    \mathcal{E}=-\frac{\mu}{2a}=\frac{v_0^2}{2}-\frac{\mu}{r_0} = \frac{v_{inf}^2}{2}
\end{equation*}
由下式计算角动量绝对值
\begin{align*}
    \vec{h} =& \vec{r}\times\vec{v} \\
    |\vec{h}| =& \vec{r}\cdot\vec{v}\cos\gamma
\end{align*}
其中,$h$为角动量,$\gamma$为进入点航迹角,
$\vec{r}$和$\vec{v}$分别为进入点位置和速度向量。
计算双曲线轨道偏心率$e$
\begin{align*}
    &a=\frac{p}{1-e^2},p=\frac{h^2}{\mu} \\
    &e=\sqrt{1-\frac{h^2}{a\mu}}
\end{align*}
其中$p$称为半通径。
计算真近点角$f$
\begin{align*}
    &r=\frac{p}{1+e\cos f} \\
    &f=\arccos(\frac{1}{e}(\frac{h^2}{\mu r}-1))
\end{align*}
由轨道元素计算位置和速度向量
\begin{align*}
    R =& \left[\begin{matrix}
        c_\Omega c_\omega-s_\Omega c_i s_\omega & -c_\Omega s_\omega-s_\Omega c_i c_\omega & s_\Omega s_i \\
        s_\Omega c_\omega+c_\Omega c_i s_\omega & -s_\Omega s_\omega+c_\Omega c_i c_\omega & -c_\Omega s_i \\
        s_i s_\omega & s_i c_\omega & c_i
    \end{matrix}\right] \\
    \vec{r} =& R\left[\begin{matrix}
        \frac{a(1-e^2)}{1+e\cos\theta}\cos\theta \\ \frac{a(1-e^2)}{1+e\cos\theta}\sin\theta \\ 0
    \end{matrix}\right] \\
    \vec{v} =& R\left[\begin{matrix}
        -\sqrt{\frac{\mu}{a(1-e^2)}}\sin\theta \\ \sqrt{\frac{\mu}{a(1-e^2)}}(e+\cos\theta) \\ 0
    \end{matrix}\right]
\end{align*}
由位置和速度向量计算远拱点
\begin{align*}
    \mathcal{E} =& \frac{\vec{v}\cdot\vec{v}}{2}
     - \frac{\mu}{|\vec{r}|} \\
    a =& -\frac{\mu}{2\mathcal{E}} \\
    \vec{e} =& \frac{\vec{v}\times\vec{h}}{\mu}
     - \frac{\vec{r}}{|\vec{r}|} \\
    e =& |\vec{e}| \\
    r_a =& a(1+e)
\end{align*}
根据平近点角计算真近点角\cite{msmart1977}
\begin{align*}
    f =& M+\left(2e-{\frac {1}{4}}e^{3}\right)\sin {M}
    + {\frac {5}{4}}e^{2}\sin {2M} \\
    &+ {\frac {13}{12}}e^{3}\sin {3M}+O(e^{4})
\end{align*}

% ////////////////////////////////////////////////////////////////
\subsection{递推最小二乘公式}
对多组数据$\vec{x}$和$y$,满足
\[y_i = \vec{x}^\mathrm{T}_i\vec{\theta}\]
其中$\vec{x}$是输入数据向量,$y$是输出数据标量。写成矩阵形式
\[\vec{y} = X\vec{\theta}\]
其中(以3组数据为例)
\begin{equation*}
    X = \left[\begin{matrix}
        \vec{x}^\mathrm{T}_1 \\
        \vec{x}^\mathrm{T}_2 \\
        \vec{x}^\mathrm{T}_3
    \end{matrix}\right]
\end{equation*}
有$k$组数据时记作
\begin{equation*}
    X^\mathrm{T}_k = \left[\begin{matrix}
        \vec{x}_1 & \vec{x}_2 & \vec{x}_3 &
        \cdots & \vec{x}_k
    \end{matrix}\right]
\end{equation*}
已知最小二乘法的解为
\begin{equation}
    \vec{\theta}
    =(X^\mathrm{T}X)^{-1}
    X^\mathrm{T}\vec{y} \label{eqFormLS}
\end{equation}
令$P = (X^\mathrm{T}X)^{-1}$,
则加上递推后
\begin{align*}
    P_k^{-1} =& X_k^\mathrm{T}X_k
    = \sum_{i=1}^k\vec{x}_i\vec{x}^\mathrm{T}_i \\
    =& \sum_{i=1}^{k-1}\vec{x}_i\vec{x}^\mathrm{T}_i
    +\vec{x}_k\vec{x}^\mathrm{T}_k \\
    =& P_{k-1}^{-1} + \vec{x}_k\vec{x}_k^\mathrm{T}
\end{align*}
同理可得
\begin{equation*}
    X_k^\mathrm{T}\vec{y}_k
    =X_{k-1}^\mathrm{T}\vec{y}_{k-1}
    +\vec{x}_ky_k
\end{equation*}
于是代入式\eqref{eqFormLS}得到
\begin{align*}
    \vec{\theta}_k =& P_kX_k^\mathrm{T}\vec{y}_k \\
    =& P_k(X_{k-1}^\mathrm{T}\vec{y}_{k-1}
    +\vec{x}_ky_k) \\
    =& P_k(P_{k-1}^{-1}\vec{\theta}_{k-1}
    +\vec{x}_ky_k) \\
    =& P_k(P_k^{-1}\vec{\theta}_{k-1}
    -\vec{x}_k\vec{x}_k^\mathrm{T}\theta_{k-1}+\vec{x}_ky_k) \\
    =& \vec{\theta}_{k-1} + P_k\vec{x}_k
    (y_k-\vec{x}_k^\mathrm{T}\theta_{k-1}) \\
    =& \vec{\theta}_{k-1}
    +(P_{k-1}^{-1} + \vec{x}_k\vec{x}_k^\mathrm{T})^{-1}
    \vec{x}_k(y_k-\vec{x}_k^\mathrm{T}\theta_{k-1}) \\
    =& \vec{\theta}_{k-1}
    +(P_{k-1}-\frac{P_{k-1}\vec{x}_k\vec{x}_k^\mathrm{T}
    P_{k-1}}{1+\vec{x}_k^\mathrm{T}P_{k-1}\vec{x}_k})
    \vec{x}_k(y_k-\vec{x}_k^\mathrm{T}\theta_{k-1}) \\
\end{align*}
其中
\begin{equation*}
    P_k = P_{k-1}-\frac{P_{k-1}\vec{x}_k\vec{x}_k^\mathrm{T}
    P_{k-1}}{1+\vec{x}_k^\mathrm{T}P_{k-1}\vec{x}_k}
\end{equation*}
用到了下面的矩阵求逆公式及其引理
\begin{align*}
    (A+BCD)^{-1} =& A^{-1}-A^{-1} B
    (DA^{-1}B+C^{-1})^{-1}DA^{-1} \\
    (A+\vec{u}\vec{u}^\text{T})^{-1} =& A^{-1}
    -\frac{A^{-1}\vec{u}\vec{u}^\text{T} A^{-1}}
    {1+\vec{u}^\text{T}A^{-1}\vec{u}}
\end{align*}
令
\begin{equation*}
    K_k = \frac{P_{k-1}\vec{x}_k}
    {1+\vec{x}_k^\mathrm{T}P_{k-1}\vec{x}_k}
\end{equation*}
则
\begin{align*}
    P_k =& (I-K_k\vec{x}_k^\mathrm{T})P_{k-1}\\
    P_k\vec{x}_k =& P_{k-1}\vec{x}_k
    -K_k\vec{x}_k^\mathrm{T}P_{k-1}\vec{x}_k \\
    =& \frac{P_{k-1}\vec{x}_k
    (1+\vec{x}_k^\mathrm{T}P_{k-1}\vec{x}_k)
    -P_{k-1}\vec{x}_k\vec{x}_k^\mathrm{T}
    P_{k-1}\vec{x}_k}
    {1+\vec{x}_k^\mathrm{T}P_{k-1}\vec{x}_k} \\
    =& \frac{P_{k-1}\vec{x}_k}
    {1+\vec{x}_k^\mathrm{T}P_{k-1}\vec{x}_k} \\
    =& K_k \\
\end{align*}
总结成递推公式得到
\begin{align}
    K_k =& \frac{P_{k-1}\vec{x}_k}{1+\vec{x}_k^\mathrm{T}
    P_{k-1}\vec{x}_k} \label{eqFormForget}\\
    P_k =& (I-K_k\vec{x}_k^\mathrm{T})P_{k-1} \notag\\
    \vec{\theta}_k =& \vec{\theta}_{k-1}
    +K_k(y_k-\vec{x}_k^\mathrm{T}\theta_{k-1}) \notag
\end{align}
若考虑遗忘因子,式\eqref{eqFormForget}改为
\begin{equation*}
    K_k = \frac{P_{k-1}\vec{x}_k}
    {\lambda+\vec{x}_k^\mathrm{T}P_{k-1}\vec{x}_k}
\end{equation*}
其中$\lambda\in(0,1]$为遗忘因子。
