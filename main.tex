\documentclass{ctacn}
\usepackage{hhline}
\usepackage{threeparttable}
\usepackage{hyperref}
\usepackage{subfigure}

\newcommand{\zhnauthora}{张航宁}
\newcommand{\zhnauthorb}{}
\newcommand{\zhnauthorc}{}
\newcommand{\zhncntitle}{火星气动捕获的建模与仿真}
\newcommand{\zhnentitle}{(Undetermined)Precision of hybrid simulations in modeling numerical solvers}
\newcommand{\zhncnabstract}{本文复现文献\cite{jqingyuan2019,dqingyuan2019}。}
\newcommand{\zhncnkeyword}{火星大气;气动捕获制导;全系数自适应;数值预测校正;鲁棒性}
\newcommand{\zhnenabstract}{This article is designed to help in the contribution for Control and Decision. It is divided into several sections. It consists of the styles and notes for the main text, the mathematical writing style and the topic of drawing tables and inserting figures respectively. The residuals deal with references, acknowledges, etc.}
\newcommand{\zhnenkeyword}{Mars atmosphere; aerocapture guidance; all-coefficient adaptive; numerical predictor-corrector; robustness}

\begin{document}

%%%%%%%
\cndoi{}
\doi{\zihao{-5}10.13195/j.kzyjc.2019.0000}
\paperdate{2018-xx-xx}{2019-xx-xx}%接收日期,修回日期
\osid{}
\setcounter{page}{1}

%%%%%%%%%%
%%%%%%%输入页眉显示的题目
%%%%%%%%%%%%
\runheading{\zhnauthora~等: \zhncntitle}%页眉设置,填写第一作者及论文题目
\xiangmujijin{本文得到国家重点研发计划(批准号:2018YFA0703800), 空间智能控制技术重点实验室基金资助(编号:ZDSYS-2018-04), 和国家自然科学基金(批准号:U20B2054)项目资助。}%项目基金  为空会自动取消显示
\authorcor{E-mail: yz37zhn@163.com.}%通讯作者邮箱,新投稿不修改
\cntitle{\zhncntitle}  % 输入中文标题
\entitle{\zhnentitle}


%                  新投稿不要修改下面的姓名及单位
%%%中文作者和单位,\dag代表通信作者,“作者一”代表3个字的名,“作者”代表2个字的名
\cnauthor{\zhnauthora\makebox{$^{1\dag}$}}  %新投稿不修改
{(中国空间技术研究院,北京100094)}  %%省会城市无需加省%%新投稿不要修改

%%%中文摘要
\cnabstract{\zhncnabstract}

%%%中文关键词
\cnkeyword{\zhncnkeyword}

%%%分类号、标识码
\clc{}  % 中文分类号,请作者自行查找并填写
\wenxianbiaoshi{A}  % 文献标志码

\citation{\zhnauthora. \zhncntitle\hspace{0pt}[J].~期刊名称,~xxxx,~xx(x):~1-xxxx.}

%%%%%%%              新投稿不要修改下面的英文名及单位
\enauthor{Hangning Zhang\makebox{$^{1\dag}$}}{
(1. China Academy of Space Technology,Beijing~100094,China)}
\enabstract{\zhnenabstract}
\enkeyword{\zhnenkeyword}
\maketitle

\begin{multicols}{2}

%%%%%%%%%%%%%%%%%%%%%%%%%%%%%%%%%%%%%%%%%%%%%%%%%%%%%%%%%%%%%%%%
% 正文
%%%%%%%%%%%%%%%%%%%%%%%%%%%%%%%%%%%%%%%%%%%%%%%%%%%%%%%%%%%%%%%%
%%%%%%%%%%%%%%%%%%%%%%%%%%%%%%%%%%%%%%%%%%%%%%%%%%%%%%%%%%%%%%%%
% 节
%%%%%%%%%%%%%%%%%%%%%%%%%%%%%%%%%%%%%%%%%%%%%%%%%%%%%%%%%%%%%%%%
\section{引\quad 言}


%%%%%%%%%%%%%%%%%%%%%%%%%%%%%%%%%%%%%%%%%%%%%%%%%%%%%%%%%%%%%%%%
% 节
%%%%%%%%%%%%%%%%%%%%%%%%%%%%%%%%%%%%%%%%%%%%%%%%%%%%%%%%%%%%%%%%
\section{实时仿真器简介}
宇宙中各天体的自转或公转周期通常以天甚至以年计算,且轨道速度快,空间尺度大。为了能够在计算机上使用微分方程求解器\cite{olzhn2021}直观地实时展示仿真结果,需要对一些公式的单位进行换算。\par
万有引力公式中,
$$\frac{\text{d}^2\vec{r}}{\text{d}t^2}=-\frac{\mu}{r^3}\vec{r}$$
中,$r$的单位是km,$t$的单位是s。因为实际的尺度太大,一个天文单位达到了$10^8$数量级,为了便于仿真,需要将实际尺度中的时间和距离变换到一个合适的时空坐标系下。
首先列举一些名词解释。实际时间:指宇宙尺度上的实际时间;无单位求解器时间:指微分方程求解器中自变量的值,没有单位,简称求解器时间,对应的速度和距离也都没有单位;仿真器消耗时间:例如仿真器exe文件运行1秒对应实际时间的1000秒,此处的1秒就是仿真器消耗时间,简称仿真器时间。现设置求解器时间的1单位时间等于实际时间的$10^4$秒,1单位距离等于实际距离$10^6$km,即在仿真软件中,
$$t_1=10^{-4}t,\ r_1=10^{-6}r$$
其中$t$和$r$分别为以秒和千米为单位的实际时间和实际距离,而$t_1$和$r_1$分别表示求解器时间和求解器距离。举个例子,1个天文单位为$r=1.5*10^8$km,则仿真软件中的1个天文单位为$r_1=150$。
然后可以计算出实际时间/速度/距离和求解器时间/速度/距离之间的一些换算关系
$$\begin{aligned}
v_1 =& \frac{\text{d}r_1}{\text{d}t_1}
 = \frac{10^{-6}\text{d}r}{10^{-4}\text{d}t} = 10^{-2}v \\
a_1 =& \frac{\text{d}}{\text{d}t_1}\frac{\text{d}r_1}{\text{d}t_1}
 = 10^2\frac{\text{d}^2r}{\text{d}t^2} \\
\mu_1(\frac{(10^{-6}\text{km})^3}{\text{kg}\cdot(10^{-4}\text{s})^2})
 =& 10^{-10}\mu(\frac{\text{km}^3}{\text{kg}\cdot \text{s}^2}) \\
-\frac{\mu_1}{r_1^2} =& -10^2\frac{\mu}{r^2} = 10^2\frac{\text{d}^2r}{\text{d}t^2} = a_1
\end{aligned}$$
可以验证,实际中的微分方程
$$a=-\frac{\mu}{r^2}$$
变换到仿真软件中仍然为
$$a_1=-\frac{\mu_1}{r_1^2}$$
若仿真步长设置为0.01求解器时间,每帧仿真10步,则实际上步长为100秒,每帧代表1000秒,按60帧计算,则仿真时间1秒等于实际时间$6\times10^4$秒,地球速度按30km/s算则仿真时间1秒内地球走过$1.8\times10^6$km,换算回仿真中则走过1.8求解器距离。
换一种方法计算,地球的求解器速度是$v_1=10^{-2}\times30=0.3$,仿真软件每1秒仿真 :60帧 $\times$ 10步/帧 $\times$ 0.01单位时间/步=6个时间单位,也可以得到仿真时间1秒内地球走过1.8求解器距离。

%%%%%%%%%%%%%%%%%%%%%%%%%%%%%%%%%%%%%%%%%%%%%%%%%%%%%%%%%%%%%%%%
% 节
%%%%%%%%%%%%%%%%%%%%%%%%%%%%%%%%%%%%%%%%%%%%%%%%%%%%%%%%%%%%%%%%
\section{轨道六根数和位置速度向量}
轨道六根数指:半长轴(a)、偏心率(e),轨道倾角(i),升交点赤经($\Omega$)、近地点幅角($omega$)、真近点角($\phi$)。根据相关公式\cite{bruiter2012}计算相关参数,已知进入轨道上某一点的速度$v_0$和该点与火星球心的距离$r_0$计算无穷远点速度
\[\begin{aligned}
    &\mathcal{E}=\frac{v_0^2}{2}-\frac{\mu}{r_0} = \frac{v_{inf}^2}{2} \\
    &v_{inf}=\sqrt{v_0^2-\frac{2\mu_m}{r}}
\end{aligned}\]
其中$\mu_m=42808$为火星的$\mu$值。计算进入火星的双曲线轨道半长轴参数$a$
\[\begin{aligned}
    &\mathcal{E}=-\frac{\mu}{2a} \\
    &a=-\frac{\mu_m}{2\mathcal{E}}=-\frac{\mu_m}{v_{inf}^2}
\end{aligned}\]
计算双曲线轨道偏心率$e$
\[\begin{aligned}
    &a=\frac{p}{1-e^2},p=\frac{h^2}{\mu} \\
    &e=\sqrt{1-\frac{h^2}{a\mu}}
\end{aligned}\]
其中$p$称为半通径,$h$为角动量大小。
计算真近点角$f$
\[\begin{aligned}
    &r=\frac{p}{1+e\cos f} \\
    &f=\arccos(\frac{1}{e}(\frac{h^2}{\mu r}-1))
\end{aligned}\]


%%%%%%%%%%%%%%%%%%%%%%%%%%%%%%%%%%%%%%%%%%%%%%%%%%%%%%%%%%%%%%%%
% 节
%%%%%%%%%%%%%%%%%%%%%%%%%%%%%%%%%%%%%%%%%%%%%%%%%%%%%%%%%%%%%%%%
\section{结\quad 论}
本文。

%%%%%%%%%%%%%%%%%%%%%%%%%%%%%%%%%%%%%%%%%%%%%%%%%%%%%%%%%%%%%%%%
% 参考文献
%%%%%%%%%%%%%%%%%%%%%%%%%%%%%%%%%%%%%%%%%%%%%%%%%%%%%%%%%%%%%%%%
\bibliographystyle{stylebib}
\bibliography{reference}


%%%%%%%%%%%%%%%%%%%%%%%%%%%%%%%%%%%%%%%%%%%%%%%%%%%%%%%%%%%%%%%%
% 作者简介,填写全部作者的简介
%%%%%%%%%%%%%%%%%%%%%%%%%%%%%%%%%%%%%%%%%%%%%%%%%%%%%%%%%%%%%%%%
\begin{authorinfo}
	\vspace{3pt}
	\zihao{-5}{\zhnauthora\,(1997$-$), 男, 博士研究生, 从事XXX、XXX等研究, E-mail: yz37zhn@163.com;}
\end{authorinfo}

\end{multicols}
\end{document}
