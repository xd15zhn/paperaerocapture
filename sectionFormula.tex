\section{基本公式推导}
下面推导常用的一些公式。

% ////////////////////////////////////////
\subsection{角速度公式}
设一向量$\vec{x}(t)$绕旋转轴$\vec{\omega}$作匀速圆周运动,
则$\vec{x}(t)$的线速度为
\[\dot{\vec{x}}(t)=\vec{\omega}\times\vec{x}(t)\]
\textbf{证明}:由罗德里格斯((Rodrigues)旋转公式
\[R=\cos\theta I+(1-\cos\theta)\vec{n}\vec{n}^\text{T}+\sin\theta\vec{n}^{\wedge}\]
其中
$R$为旋转矩阵,
$n$为单位旋转轴,
$\theta$为旋转角度,
$\vec{n}^{\wedge}$表示向量$\vec{n}$叉乘对应的反对称矩阵。
将向量$\vec{\omega}$写成$\vec{\omega}=\omega\vec{n}$,
并将旋转矩阵对时间$t$求导得到
\begin{align*}
    \frac{\text{d}}{\text{d}t}R(t)
    =& \frac{\text{d}}{\text{d}t}\left(
        \cos\omega t I
        +(1-\cos\omega t)\frac{\vec{\omega}}{\omega}\frac{\vec{\omega}^\text{T}}{\omega}
        +\sin\omega t\frac{\vec{\omega}^{\wedge}}{\omega}
    \right) \\
    =& -\omega\sin\omega tI
        +\frac{\sin\omega t}{\omega}\vec{\omega}\vec{\omega}^\text{T}
        +\cos\omega t\cdot\vec{\omega}^{\wedge} \\
    \dot{R}(t)\vec{x}_0
    =& -\omega\sin\omega t\vec{x}_0
        +\frac{\sin\omega t}{\omega}\vec{\omega}^\text{T}\vec{x}_0\vec{\omega}
        +\cos\omega t\cdot\vec{\omega}\times\vec{x}_0
\end{align*}
另因为
\begin{align*}
    \vec{\omega}\times R(t)\vec{x}_0
    =& \cos\omega t\cdot\vec{\omega}\times\vec{x}_0
        +\frac{1-\cos\omega t}{\omega^2}\vec{\omega}\times\vec{\omega}\vec{\omega}^\text{T}\vec{x}_0 \\
        &+ \frac{\sin\omega t}{\omega}\vec{\omega}\times(\vec{\omega}\times\vec{x}_0) \\
    =& \cos\omega t\cdot\vec{\omega}\times\vec{x}_0
        +\frac{\sin\omega t}{\omega}(\vec{\omega}^\text{T}\vec{x}_0\vec{\omega}
        -\vec{\omega}^\text{T}\vec{\omega}\vec{x}_0)
\end{align*}
所以
\[\dot{\vec{x}}(t)=\dot{R}(t)\vec{x}_0=\vec{\omega}\times R(t)\vec{x}_0=\vec{\omega}\times\vec{x}(t)\]

% ////////////////////////////////////////
\subsection{旋转坐标系下的速度和加速度}
旋转坐标系$\mathcal{F}_2$绕惯性坐标系$\mathcal{F}_1$以角速度$\vec{\omega}$旋转,
设$\mathcal{F}_1$下位置向量的一阶导和二阶导分别为
$\dot{\vec{r}}$和$\ddot{\vec{r}}$,
$\mathcal{F}_2$下位置向量的一阶导和二阶导为
$\overset{\circ}{\vec{r}}$和$\overset{\circ\circ}{\vec{r}}$,
满足
\begin{align*}
    \dot{\vec{r}}
    =& \overset{\circ}{\vec{r}}
    + \vec{\omega}\times\vec{r} \\
    \ddot{\vec{r}}
    =& \overset{\circ\circ}{\vec{r}}
    + 2\vec{\omega}\times\overset{\circ}{\vec{r}}
    + \overset{\circ}{\vec{\omega}}\times\vec{r}
    + \vec{\omega}\times(\vec{\omega}\times\vec{r})
\end{align*}
\textbf{证明}:
设同一向量$\vec{r}$在坐标系$\mathcal{F}_1$和$\mathcal{F}_2$下的坐标分别为
\begin{equation*}
    \vec{r} = \left[\begin{matrix}
        \vec{e}_x & \vec{e}_y & \vec{e}_z
    \end{matrix}\right]
    \left[\begin{matrix}
        x_1 \\ y_1 \\ z_1
    \end{matrix}\right]
    = \left[\begin{matrix}
        \vec{a}_x & \vec{a}_y & \vec{a}_z
    \end{matrix}\right]
    \left[\begin{matrix}
        x_2 \\ y_2 \\ z_2
    \end{matrix}\right]
\end{equation*}
其中$[\vec{e}_x\ \vec{e}_y\ \vec{e}_z]$表示惯性坐标系$\mathcal{F}_1$下的三轴单位向量,
$[\vec{a}_x\ \vec{a}_y\ \vec{a}_z]$表示坐标系$\mathcal{F}_2$的三轴单位向量在惯性坐标系$\mathcal{F}_1$下的坐标,
三个单位向量张成旋转坐标系$\mathcal{F}_2$,
则向量$\vec{r}$的一阶导
\begin{align*}
    \dot{\vec{r}}
    =& \left[\begin{matrix}
        \dot{\vec{a}}_x & \dot{\vec{a}}_y & \dot{\vec{a}}_z
    \end{matrix}\right]
    \left[\begin{matrix}
        x_2 \\ y_2 \\ z_2
    \end{matrix}\right]
    + \left[\begin{matrix}
        \vec{a}_x & \vec{a}_y & \vec{a}_z
    \end{matrix}\right]
    \left[\begin{matrix}
        \dot{x}_2 \\ \dot{y}_2 \\ \dot{z}_2
    \end{matrix}\right] \\
    =& \vec{\omega}\times
    \left[\begin{matrix}
        \vec{a}_x & \vec{a}_y & \vec{a}_z
    \end{matrix}\right]
    \left[\begin{matrix}
        x_2 \\ y_2 \\ z_2
    \end{matrix}\right]
    + \overset{\circ}{\vec{r}} \\
    =& \vec{\omega}\times\vec{r} + \overset{\circ}{\vec{r}}
\end{align*}
使用坐标系的记法写作
\begin{align*}
    \dot{\vec{r}}
    =& \frac{\text{d}}{\text{d}t}(\mathcal{F}_2\vec{r}_2) \\
    =& \mathcal{F}_2\dot{\vec{r}}_2
    + \dot{\mathcal{F}_2}\vec{r}_2 \\
    =& \mathcal{F}_2\dot{\vec{r}}_2
    + \vec{\omega} \times \mathcal{F}_2\vec{r}_2 \\
    =& \mathcal{F}_2\dot{\vec{r}}_2
    + \vec{\omega} \times \mathcal{F}_1\vec{r}_1
\end{align*}
对上式进一步求导得
\begin{align*}
    \ddot{\vec{r}}
    =& \vec{\omega} \times \mathcal{F}_2\dot{\vec{r}}_2
    + \mathcal{F}_2\ddot{\vec{r}}_2
    + \dot{\vec{\omega}} \times \mathcal{F}_1\vec{r}_1 \\
    &+ \vec{\omega} \times (\vec{\omega} \times \mathcal{F}_2\vec{r}_2
    + \mathcal{F}_2\dot{\vec{r}}_2) \\
    =& \mathcal{F}_2\ddot{\vec{r}}_2
    + 2\vec{\omega} \times \mathcal{F}_2\dot{\vec{r}}_2 \\
    &+ \dot{\vec{\omega}} \times \mathcal{F}_1\vec{r}_1
    + \vec{\omega} \times (\vec{\omega} \times \mathcal{F}_1\vec{r}_1) \\
    =& \overset{\circ\circ}{\vec{r}}
    + 2\vec{\omega}\times\overset{\circ}{\vec{r}}
    + \dot{\vec{\omega}}\times\vec{r}
    + \vec{\omega}\times(\vec{\omega}\times\vec{r})
\end{align*}
其中$\vec{\omega}$在两个坐标系下的坐标相等,
即$\dot{\vec{\omega}}=\overset{\circ}{\vec{\omega}}$。

% ////////////////////////////////////////////////////////////////
\subsection{轨道六根数和位置速度向量}
轨道六根数指:
半长轴(a)、偏心率(e),轨道倾角(i),升交点赤经($\Omega$)、近地点幅角($\omega$)、真近点角($\phi$)。
根据文献\cite{mruiter2012}中的公式计算相关参数。
根据平近点角计算真近点角\cite{msmart1977}
\begin{align*}
    f =& M+\left(2e-{\frac {1}{4}}e^{3}\right)\sin {M}
    + {\frac {5}{4}}e^{2}\sin {2M} \\
    &+ {\frac {13}{12}}e^{3}\sin {3M}+O(e^{4})
\end{align*}
