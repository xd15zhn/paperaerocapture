\section{预测校正制导}
理论上当大气入口条件和变轨目标确定后,
即可得出一条确定轨迹,气动飞行期间不用施加控制即可实现既定的捕获任务。
但实际飞行中存在极强的环境不确定性,飞行器及进入轨道本身也存在初始条件的不确定性,因此需要设计在轨制导律。
气动飞行的制导方法主要分为标称轨迹制导和预测校正制导。
制导算法的鲁棒性也是需要重点考虑的问题关键,在不确定性强的环境下,
预测校正制导相对标称轨迹的解析制导更具备优势\cite{dqingyuan2019}。
在轨制导律的控制量输出是倾侧角随时间变化的函数,
如果控制器离散,则输出的是倾侧角序列。
原文使用的是离散的全系数自适应控制。

原文中提出了一种标称制导和两种预测校正制导方法。
标称制导指的是在假设没有各种随机因素影响的情况下,
以总速度增量最小或能量最小为目标的最优控制,
生成的倾侧角序列为先以最小倾侧角飞行,
再以最大倾侧角飞行,此时能量最优。
两种预测校正制导方法的控制律相同,但制导目标不同。
一种是目标远拱点误差,也就是以实际远点高度与设定的远点高度的差作为评价指标;
另一种是远拱点抬升速度增量,也就是在近拱点处开启发动机调整远拱点时,以速度增量大小作为评价指标。
本文复现第一种目标。

制导部分被控对象的输入是倾侧角,输出是远拱点误差。
在离散控制器的每个更新时刻,计算如果保持当前输入的倾侧角指令不变,
飞出大气层后以当前的位置和速度向量换算成轨道元素,进而求出远拱点误差。

被控对象的特征模型为
\begin{align*}
    &y(k) = \Phi^{\text{T}}(k)\theta(k) + e(k) \\
    &\theta(k) = \theta(k-1) + 
\end{align*}
预测校正制导的控制律使用线性反馈控制:
\begin{align*}
    &u(k) = u(k-1) + u_L(k) \\
    &u_L(k) = -\frac{l_1\alpha_1(k)y(k)}{\beta_0(k)}
\end{align*}
第一种制导目标中,
\begin{equation*}
    y(k) = r_a^e(k) - r_a^*
\end{equation*}

