%%%%%%%%%%%%%%%%%%%%%%%%%%%%%%%%%%%%%%%%%%%%%%%%%%%%%%%%%%%%%%%%
% Real
%%%%%%%%%%%%%%%%%%%%%%%%%%%%%%%%%%%%%%%%%%%%%%%%%%%%%%%%%%%%%%%%
\section{实时仿真理论}
宇宙中各天体的自转或公转周期通常以天甚至以年计算,
且轨道速度快,空间尺度大。
如果按照实际参数进行仿真,
则仿真的计算量太大而无法实时计算。
为了能够在仿真时直观地实时展示仿真结果,
需要对一些公式的单位进行换算。

万有引力公式
\begin{equation}
    \frac{\text{d}^2\vec{r}}{\text{d}t^2}=-\frac{\mu}{r^3}\vec{r} \label{eqRealGravity}
\end{equation}
中,$r$的单位是km,$t$的单位是s。
因为实际的尺度太大,一个天文单位达到了$10^8$数量级,
为了便于仿真,
需要将实际尺度中的时间和距离变换到一个合适的时空坐标系下。
首先列举一些名词解释。
\textbf{实际时间}指宇宙尺度上的实际时间;
\textbf{无单位求解器时间}指微分方程求解器中自变量的值,
没有单位,简称求解器时间,
对应的求解器速度和求解器距离也都没有单位;
\textbf{仿真展示软件消耗时间}:例如仿真展示软件运行1秒对应实际时间的1000秒,
此处的1秒指的是是仿真展示软件消耗时间,简称展示时间。

举例说明,
求解器时间的1单位时间等于实际时间的$10^4$秒,
1单位距离等于实际距离$10^6$km,即在求解器中,
\begin{equation}
    t_1=10^{-4}t,\ r_1=10^{-6}r \label{eqRealConvert}
\end{equation}
其中$t$和$r$分别为以秒和千米为单位的实际时间和实际距离,
而$t_1$和$r_1$分别表示求解器时间和求解器距离。
1个天文单位为$r=1.5*10^8$km,
则求解器中的1个天文单位为$r_1=150$。
然后可以计算出实际时间/速度/距离
和求解器时间/速度/距离之间的一些换算关系
\begin{align*}
v_1 =& \frac{\text{d}r_1}{\text{d}t_1}
 = \frac{10^{-6}\text{d}r}{10^{-4}\text{d}t} = 10^{-2}v \\
a_1 =& \frac{\text{d}}{\text{d}t_1}\frac{\text{d}r_1}{\text{d}t_1}
 = 10^2\frac{\text{d}^2r}{\text{d}t^2} \\
\mu_1(\frac{(10^{-6}\text{km})^3}{\text{kg}\cdot(10^{-4}\text{s})^2})
 =& 10^{-10}\mu(\frac{\text{km}^3}{\text{kg}\cdot \text{s}^2}) \\
-\frac{\mu_1}{r_1^2} =& -10^2\frac{\mu}{r^2} = 10^2\frac{\text{d}^2r}{\text{d}t^2} = a_1
\end{align*}
可以验证,实际中的微分方程
$$a=-\frac{\mu}{r^2}$$
变换到求解器中仍然为
$$a_1=-\frac{\mu_1}{r_1^2}$$

下面在换算公式\eqref{eqRealConvert}的基础上引入仿真展示软件消耗时间。
若仿真步长设置为0.01求解器时间,仿真展示软件每帧仿真10步,
则换算到实际时间上的步长为100秒,每帧代表1000秒,
按60帧计算,则展示时间1秒等于实际时间$6\times10^4$秒,
地球速度按30km/s算则展示时间1秒内地球走过$1.8\times10^6$km,
换算回仿真中则走过1.8展示距离。
换一种方法计算,
地球的求解器速度是$v_1=10^{-2}\times30=0.3$,
仿真展示软件每1秒仿真:
60帧$\times$10步/帧$\times$0.01单位时间/步=6个时间单位,
也可以得到展示时间1秒内地球走过1.8展示距离。
在式\eqref{eqRealConvert}的设定下,
仿真展示软件每帧仿真1步与10步对应的地球自转周期分别约为$14.4$和$1.44$秒,
分别可用于展示低轨卫星和地球同步卫星的动态运行结果。

与实时仿真不同的是实际中广泛使用的非实时仿真,
也就是在一次仿真结束后展示静态仿真结果。
此时虽然不需要考虑展示时间/速度/距离,
但求解器时间/速度/距离仍然需要考虑。
针对本文的火星气动捕获场景,
将换算公式\eqref{eqRealConvert}改为,
\begin{equation}
    t_1=10^{-4}t,\ r_1=10^{-3}r
\end{equation}
则。

如果仿真需要展示系统运动的动态过程,
则应使用实时仿真,否则应使用非实时仿真。
本文中火星气动捕获制导结果仅需要展示最终制导轨迹,
因此使用使用非实时仿真。
