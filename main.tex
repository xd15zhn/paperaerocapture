\documentclass{ctacn}
\usepackage{hhline}
\usepackage{threeparttable}
\usepackage{hyperref}
\usepackage{subfigure}

\newcommand{\zhnauthora}{张航宁}
\newcommand{\zhnauthorb}{}
\newcommand{\zhnauthorc}{}
\newcommand{\zhncntitle}{火星气动捕获的建模、控制与仿真}
\newcommand{\zhnentitle}{(Undetermined)Precision of hybrid simulations in modeling numerical solvers}
\newcommand{\zhncnabstract}{本文复现文献\cite{dqingyuan2019},公开版本见\cite{jqingyuan2019guidance,jqingyuan2019coefficient,jqingyuan2019observer}。}
\newcommand{\zhncnkeyword}{火星大气;气动捕获制导;全系数自适应;数值预测校正;鲁棒性}
\newcommand{\zhnenabstract}{This article is designed to help in the contribution for Control and Decision. It is divided into several sections. It consists of the styles and notes for the main text, the mathematical writing style and the topic of drawing tables and inserting figures respectively. The residuals deal with references, acknowledges, etc.}
\newcommand{\zhnenkeyword}{Mars atmosphere; aerocapture guidance; all-coefficient adaptive; numerical predictor-corrector; robustness}

\begin{document}

%%%%%%%
\cndoi{}
\doi{\zihao{-5}10.13195/j.kzyjc.2019.0000}
\paperdate{2018-xx-xx}{2019-xx-xx}%接收日期,修回日期
\osid{}
\setcounter{page}{1}

%%%%%%%%%%
%%%%%%%输入页眉显示的题目
%%%%%%%%%%%%
\runheading{\zhnauthora~等: \zhncntitle}%页眉设置,填写第一作者及论文题目
\xiangmujijin{本文得到国家重点研发计划(批准号:2018YFA0703800), 空间智能控制技术重点实验室基金资助(编号:ZDSYS-2018-04), 和国家自然科学基金(批准号:U20B2054)项目资助。}%项目基金  为空会自动取消显示
\authorcor{E-mail: yz37zhn@163.com.}%通讯作者邮箱,新投稿不修改
\cntitle{\zhncntitle}  % 输入中文标题
\entitle{\zhnentitle}


%                  新投稿不要修改下面的姓名及单位
%%%中文作者和单位,\dag代表通信作者,“作者一”代表3个字的名,“作者”代表2个字的名
\cnauthor{\zhnauthora\makebox{$^{1\dag}$}}  %新投稿不修改
{(北京控制工程研究所,北京100094)}  %%省会城市无需加省%%新投稿不要修改

%%%中文摘要
\cnabstract{\zhncnabstract}

%%%中文关键词
\cnkeyword{\zhncnkeyword}

%%%分类号、标识码
\clc{}  % 中文分类号,请作者自行查找并填写
\wenxianbiaoshi{A}  % 文献标志码

\citation{\zhnauthora. \zhncntitle\hspace{0pt}[J].~期刊名称,~xxxx,~xx(x):~1-xxxx.}

%%%%%%%              新投稿不要修改下面的英文名及单位
\enauthor{Hangning Zhang\makebox{$^{1\dag}$}}{
(1. China Academy of Space Technology,Beijing~100094,China)}
\enabstract{\zhnenabstract}
\enkeyword{\zhnenkeyword}
\maketitle

\begin{multicols}{2}

%%%%%%%%%%%%%%%%%%%%%%%%%%%%%%%%%%%%%%%%%%%%%%%%%%%%%%%%%%%%%%%%
% 正文
%%%%%%%%%%%%%%%%%%%%%%%%%%%%%%%%%%%%%%%%%%%%%%%%%%%%%%%%%%%%%%%%
\section{引\quad 言}
气动捕获制导的总体思路是,
通过不断预测以当前控制量飞出大气后的过渡轨道状态向量与目标轨道的状态向量间的误差,
代入制导算法来产生新的控制量,
如此往复,直到满足终端需求为止。

气动捕获采用倾侧角作为控制量而不是攻角的原因是,
攻角幅度过大会影响飞行器的热控能力。

\section{基本公式推导}
下面推导常用的一些公式。其中
角速度公式和相对加速度公式用于飞行器再入动力学方程,
递推最小二乘公式用于基于特征模型的全系数自适应控制。

% ////////////////////////////////////////
\subsection{角速度公式}
设一向量$\vec{x}(t)$绕旋转轴$\vec{\omega}$作匀速圆周运动,
则$\vec{x}(t)$的线速度为
\[\dot{\vec{x}}(t)=\vec{\omega}\times\vec{x}(t)\]
\textbf{证明}:由罗德里格斯((Rodrigues)旋转公式
\[R=\cos\theta I+(1-\cos\theta)\vec{n}\vec{n}^\text{T}+\sin\theta\vec{n}^{\wedge}\]
其中
$R$为旋转矩阵,
$n$为单位旋转轴,
$\theta$为旋转角度,
$\vec{n}^{\wedge}$表示向量$\vec{n}$叉乘对应的反对称矩阵。
将向量$\vec{\omega}$写成$\vec{\omega}=\omega\vec{n}$,
并将旋转矩阵对时间$t$求导得到
\begin{align*}
    \frac{\text{d}}{\text{d}t}R(t)
    =& \frac{\text{d}}{\text{d}t}\left(
        \cos\omega t I
        +(1-\cos\omega t)\frac{\vec{\omega}}{\omega}\frac{\vec{\omega}^\text{T}}{\omega}
        +\sin\omega t\frac{\vec{\omega}^{\wedge}}{\omega}
    \right) \\
    =& -\omega\sin\omega tI
        +\frac{\sin\omega t}{\omega}\vec{\omega}\vec{\omega}^\text{T}
        +\cos\omega t\cdot\vec{\omega}^{\wedge} \\
    \dot{R}(t)\vec{x}_0
    =& -\omega\sin\omega t\vec{x}_0
        +\frac{\sin\omega t}{\omega}\vec{\omega}^\text{T}\vec{x}_0\vec{\omega}
        +\cos\omega t\cdot\vec{\omega}\times\vec{x}_0
\end{align*}
另因为
\begin{align*}
    \vec{\omega}\times R(t)\vec{x}_0
    =& \cos\omega t\cdot\vec{\omega}\times\vec{x}_0
        +\frac{1-\cos\omega t}{\omega^2}\vec{\omega}\times\vec{\omega}\vec{\omega}^\text{T}\vec{x}_0 \\
        &+ \frac{\sin\omega t}{\omega}\vec{\omega}\times(\vec{\omega}\times\vec{x}_0) \\
    =& \cos\omega t\cdot\vec{\omega}\times\vec{x}_0
        +\frac{\sin\omega t}{\omega}(\vec{\omega}^\text{T}\vec{x}_0\vec{\omega}
        -\vec{\omega}^\text{T}\vec{\omega}\vec{x}_0)
\end{align*}
所以
\[\dot{\vec{x}}(t)=\dot{R}(t)\vec{x}_0=\vec{\omega}\times R(t)\vec{x}_0=\vec{\omega}\times\vec{x}(t)\]

% ////////////////////////////////////////
\subsection{旋转坐标系下的速度和加速度}
旋转坐标系$\mathcal{F}_2$绕惯性坐标系$\mathcal{F}_1$以角速度$\vec{\omega}$旋转,
设$\mathcal{F}_1$下位置向量的一阶导和二阶导分别为
$\dot{\vec{r}}$和$\ddot{\vec{r}}$,
$\mathcal{F}_2$下位置向量的一阶导和二阶导为
$\overset{\circ}{\vec{r}}$和$\overset{\circ\circ}{\vec{r}}$,
满足
\begin{align*}
    \dot{\vec{r}}
    =& \overset{\circ}{\vec{r}}
    + \vec{\omega}\times\vec{r} \\
    \ddot{\vec{r}}
    =& \overset{\circ\circ}{\vec{r}}
    + 2\vec{\omega}\times\overset{\circ}{\vec{r}}
    + \overset{\circ}{\vec{\omega}}\times\vec{r}
    + \vec{\omega}\times(\vec{\omega}\times\vec{r})
\end{align*}
\textbf{证明}:
设同一向量$\vec{r}$在坐标系$\mathcal{F}_1$和$\mathcal{F}_2$下的坐标分别为
\begin{equation*}
    \vec{r} = \left[\begin{matrix}
        \vec{e}_x & \vec{e}_y & \vec{e}_z
    \end{matrix}\right]
    \left[\begin{matrix}
        x_1 \\ y_1 \\ z_1
    \end{matrix}\right]
    = \left[\begin{matrix}
        \vec{a}_x & \vec{a}_y & \vec{a}_z
    \end{matrix}\right]
    \left[\begin{matrix}
        x_2 \\ y_2 \\ z_2
    \end{matrix}\right]
\end{equation*}
其中$[\vec{e}_x\ \vec{e}_y\ \vec{e}_z]$表示惯性坐标系$\mathcal{F}_1$下的三轴单位向量,
$[\vec{a}_x\ \vec{a}_y\ \vec{a}_z]$表示坐标系$\mathcal{F}_2$的三轴单位向量在惯性坐标系$\mathcal{F}_1$下的坐标,
三个单位向量张成旋转坐标系$\mathcal{F}_2$,
则向量$\vec{r}$的一阶导
\begin{align*}
    \dot{\vec{r}}
    =& \left[\begin{matrix}
        \dot{\vec{a}}_x & \dot{\vec{a}}_y & \dot{\vec{a}}_z
    \end{matrix}\right]
    \left[\begin{matrix}
        x_2 \\ y_2 \\ z_2
    \end{matrix}\right]
    + \left[\begin{matrix}
        \vec{a}_x & \vec{a}_y & \vec{a}_z
    \end{matrix}\right]
    \left[\begin{matrix}
        \dot{x}_2 \\ \dot{y}_2 \\ \dot{z}_2
    \end{matrix}\right] \\
    =& \vec{\omega}\times
    \left[\begin{matrix}
        \vec{a}_x & \vec{a}_y & \vec{a}_z
    \end{matrix}\right]
    \left[\begin{matrix}
        x_2 \\ y_2 \\ z_2
    \end{matrix}\right]
    + \overset{\circ}{\vec{r}} \\
    =& \vec{\omega}\times\vec{r} + \overset{\circ}{\vec{r}}
\end{align*}
使用坐标系的记法写作
\begin{align*}
    \dot{\vec{r}}
    =& \frac{\text{d}}{\text{d}t}(\mathcal{F}_2\vec{r}_2) \\
    =& \mathcal{F}_2\dot{\vec{r}}_2
    + \dot{\mathcal{F}_2}\vec{r}_2 \\
    =& \mathcal{F}_2\dot{\vec{r}}_2
    + \vec{\omega} \times \mathcal{F}_2\vec{r}_2 \\
    =& \mathcal{F}_2\dot{\vec{r}}_2
    + \vec{\omega} \times \mathcal{F}_1\vec{r}_1
\end{align*}
对上式进一步求导得
\begin{align*}
    \ddot{\vec{r}}
    =& \vec{\omega} \times \mathcal{F}_2\dot{\vec{r}}_2
    + \mathcal{F}_2\ddot{\vec{r}}_2
    + \dot{\vec{\omega}} \times \mathcal{F}_1\vec{r}_1 \\
    &+ \vec{\omega} \times (\vec{\omega} \times \mathcal{F}_2\vec{r}_2
    + \mathcal{F}_2\dot{\vec{r}}_2) \\
    =& \mathcal{F}_2\ddot{\vec{r}}_2
    + 2\vec{\omega} \times \mathcal{F}_2\dot{\vec{r}}_2 \\
    &+ \dot{\vec{\omega}} \times \mathcal{F}_1\vec{r}_1
    + \vec{\omega} \times (\vec{\omega} \times \mathcal{F}_1\vec{r}_1) \\
    =& \overset{\circ\circ}{\vec{r}}
    + 2\vec{\omega}\times\overset{\circ}{\vec{r}}
    + \dot{\vec{\omega}}\times\vec{r}
    + \vec{\omega}\times(\vec{\omega}\times\vec{r})
\end{align*}
其中$\vec{\omega}$在两个坐标系下的坐标相等,
即$\dot{\vec{\omega}}=\overset{\circ}{\vec{\omega}}$。

% ////////////////////////////////////////////////////////////////
\subsection{轨道六根数和位置速度向量} \label{secFormElement}
轨道六根数指:
半长轴(a)、偏心率(e),轨道倾角(i),升交点赤经($\Omega$)、近地点幅角($\omega$)、真近点角($\phi$)。
根据文献\cite{mruiter2012}中的公式计算相关参数。
由下式计算轨道能量
\begin{equation*}
    \mathcal{E}=-\frac{\mu}{2a}=\frac{v_0^2}{2}-\frac{\mu}{r_0} = \frac{v_{inf}^2}{2}
\end{equation*}
由下式计算角动量绝对值
\begin{align*}
    \vec{h} =& \vec{r}\times\vec{v} \\
    |\vec{h}| =& \vec{r}\cdot\vec{v}\cos\gamma
\end{align*}
其中,$h$为角动量,$\gamma$为进入点航迹角,
$\vec{r}$和$\vec{v}$分别为进入点位置和速度向量。
计算双曲线轨道偏心率$e$
\begin{align*}
    &a=\frac{p}{1-e^2},p=\frac{h^2}{\mu} \\
    &e=\sqrt{1-\frac{h^2}{a\mu}}
\end{align*}
其中$p$称为半通径。
计算真近点角$f$
\begin{align*}
    &r=\frac{p}{1+e\cos f} \\
    &f=\arccos(\frac{1}{e}(\frac{h^2}{\mu r}-1))
\end{align*}
由轨道元素计算位置和速度向量
\begin{align}
    R =& \left[\begin{matrix}
        c_\Omega c_\omega-s_\Omega c_i s_\omega & -c_\Omega s_\omega-s_\Omega c_i c_\omega & s_\Omega s_i \\
        s_\Omega c_\omega+c_\Omega c_i s_\omega & -s_\Omega s_\omega+c_\Omega c_i c_\omega & -c_\Omega s_i \\
        s_i s_\omega & s_i c_\omega & c_i
    \end{matrix}\right] \notag\\
    \vec{r} =& R\left[\begin{matrix}
        \frac{a(1-e^2)}{1+e\cos\theta}\cos\theta \\ \frac{a(1-e^2)}{1+e\cos\theta}\sin\theta \\ 0
    \end{matrix}\right] \notag\\
    \vec{v} =& R\left[\begin{matrix}
        -\sqrt{\frac{\mu}{a(1-e^2)}}\sin\theta \\ \sqrt{\frac{\mu}{a(1-e^2)}}(e+\cos\theta) \\ 0
    \end{matrix}\right] \label{eqFormEle2RV}
\end{align}
根据平近点角计算真近点角\cite{msmart1977}
\begin{align*}
    f =& M+\left(2e-{\frac {1}{4}}e^{3}\right)\sin {M}
    + {\frac {5}{4}}e^{2}\sin {2M} \\
    &+ {\frac {13}{12}}e^{3}\sin {3M}+O(e^{4})
\end{align*}

% ////////////////////////////////////////////////////////////////
\subsection{递推最小二乘公式}
对多组数据$\boldsymbol{x}$和$y$,满足
\[y_i = \boldsymbol{x}^\mathrm{T}_i\boldsymbol{\theta}\]
其中$\boldsymbol{x}$是输入数据向量,$y$是输出数据标量。写成矩阵形式
\[\boldsymbol{y} = \mathbf{X}\theta\]
其中
\begin{align*}
    \mathbf{X} =& \left[\begin{matrix}
        \boldsymbol{x}^\mathrm{T}_1 \\
        \boldsymbol{x}^\mathrm{T}_2 \\
        \boldsymbol{x}^\mathrm{T}_3
    \end{matrix}\right] \\
    \mathbf{X}^\mathrm{T}_k =& \left[\begin{matrix}
        \boldsymbol{x}_1 & \boldsymbol{x}_2 & \boldsymbol{x}_3 &
        \cdots & \boldsymbol{x}_k
    \end{matrix}\right]
\end{align*}
最小二乘解为
\begin{equation*}
    \boldsymbol{\theta}
    =(\mathbf{X}^\mathrm{T}\mathbf{X})^{-1}
    \mathbf{X}^\mathrm{T}\boldsymbol{y}\tag{2.1}
\end{equation*}
令
\begin{align*}
    \mathbf{P}^{-1} =& \mathbf{X}^\mathrm{T}\mathbf{X} \\
    \mathbf{P}_k^{-1} =& \sum_{i=1}^k\boldsymbol{x}_i\boldsymbol{x}^\mathrm{T}_i
    = \sum_{i=1}^{k-1}\boldsymbol{x}_i\boldsymbol{x}^\mathrm{T}_i
    +\boldsymbol{x}_k\boldsymbol{x}^\mathrm{T}_k \\
    =& \mathbf{P}_{k-1}^{-1} + \boldsymbol{x}_k\boldsymbol{x}_k^\mathrm{T}
\end{align*}
同理可得
\begin{equation*}
    \mathbf{X}_k^\mathrm{T}\boldsymbol{y}_k
    =\mathbf{X}_{k-1}^\mathrm{T}\boldsymbol{y}_{k-1}
    +\boldsymbol{x}_ky_k
\end{equation*}
于是代入式(1)得到
\begin{align*}
\boldsymbol{\theta}_k =& \mathbf{P}_k\mathbf{X}_k^\mathrm{T}\boldsymbol{y}_k \\
=& \mathbf{P}_k(\mathbf{X}_{k-1}^\mathrm{T}\boldsymbol{y}_{k-1}
 +\boldsymbol{x}_ky_k) \\
=& \mathbf{P}_k(\mathbf{P}_{k-1}^{-1}\boldsymbol{\theta}_{k-1}
 +\boldsymbol{x}_ky_k) \\
=& \mathbf{P}_k(\mathbf{P}_k^{-1}\boldsymbol{\theta}_{k-1}
 -\boldsymbol{x}_k\boldsymbol{x}_k^\mathrm{T}\theta_{k-1}+\boldsymbol{x}_ky_k) \\
=& \boldsymbol{\theta}_{k-1} + \mathbf{P}_k\boldsymbol{x}_k
 (y_k-\boldsymbol{x}_k^\mathrm{T}\theta_{k-1}) \\
=& \boldsymbol{\theta}_{k-1}
 +(\mathbf{P}_{k-1}^{-1} + \boldsymbol{x}_k\boldsymbol{x}_k^\mathrm{T})^{-1}
 \boldsymbol{x}_k(y_k-\boldsymbol{x}_k^\mathrm{T}\theta_{k-1}) \\
=& \boldsymbol{\theta}_{k-1}
 +(\mathbf{P}_{k-1}-\frac{\mathbf{P}_{k-1}\boldsymbol{x}_k\boldsymbol{x}_k^\mathrm{T}
 \mathbf{P}_{k-1}}{1+\boldsymbol{x}_k^\mathrm{T}\mathbf{P}_{k-1}\boldsymbol{x}_k})
 \boldsymbol{x}_k(y_k-\boldsymbol{x}_k^\mathrm{T}\theta_{k-1}) \\
\end{align*}
其中
\begin{equation*}
    \mathbf{P}_k =
 \mathbf{P}_{k-1}-\frac{\mathbf{P}_{k-1}\boldsymbol{x}_k\boldsymbol{x}_k^\mathrm{T}
 \mathbf{P}_{k-1}}{1+\boldsymbol{x}_k^\mathrm{T}\mathbf{P}_{k-1}\boldsymbol{x}_k}
\end{equation*}
用到了下面的矩阵求逆公式及其引理
\begin{align*}
(\mathbf{A}+\mathbf{B C D})^{-1} =& \mathbf{A}^{-1}-\mathbf{A}^{-1} \mathbf{B}
(\mathbf{D A} \mathbf{A}^{-1} \mathbf{B}+\mathbf{C}^{-1})^{-1} \mathbf{D A}^{-1} \\
(\mathbf{A}+\mathbf{u} \mathbf{u}^{T})^{-1} =& \mathbf{A}^{-1}
 -\frac{\mathbf{A}^{-1} \mathbf{u} \mathbf{u}^{T} \mathbf{A}^{-1}}
 {1+\mathbf{u}^{T} \mathbf{A}^{-1} \mathbf{u}}
\end{align*}
令
\begin{equation*}
    \mathbf{K}_k = \frac{\mathbf{P}_{k-1}\boldsymbol{x}_k}
{1+\boldsymbol{x}_k^\mathrm{T}\mathbf{P}_{k-1}\boldsymbol{x}_k}
\end{equation*}
则
\begin{align*}
\mathbf{P}_k =& (\mathbf{I}-\mathbf{K}_k\boldsymbol{x}_k^\mathrm{T})\mathbf{P}_{k-1}\\
\mathbf{P}_k\boldsymbol{x}_k =& \mathbf{P}_{k-1}\boldsymbol{x}_k
 -\mathbf{K}_k\boldsymbol{x}_k^\mathrm{T}\mathbf{P}_{k-1}\boldsymbol{x}_k \\
=& \frac{\mathbf{P}_{k-1}\boldsymbol{x}_k
 (1+\boldsymbol{x}_k^\mathrm{T}\mathbf{P}_{k-1}\boldsymbol{x}_k)
 -\mathbf{P}_{k-1}\boldsymbol{x}_k\boldsymbol{x}_k^\mathrm{T}
 \mathbf{P}_{k-1}\boldsymbol{x}_k}
 {1+\boldsymbol{x}_k^\mathrm{T}\mathbf{P}_{k-1}\boldsymbol{x}_k} \\
=& \frac{\mathbf{P}_{k-1}\boldsymbol{x}_k}
 {1+\boldsymbol{x}_k^\mathrm{T}\mathbf{P}_{k-1}\boldsymbol{x}_k} \\
=& \mathbf{K}_k \\
\end{align*}
总结成递推公式得到
\begin{align*}
\mathbf{K}_k =& \frac{\mathbf{P}_{k-1}\boldsymbol{x}_k}{1+\boldsymbol{x}_k^\mathrm{T}
\mathbf{P}_{k-1}\boldsymbol{x}_k} \\
\mathbf{P}_k =& (\mathbf{I}-\mathbf{K}_k\boldsymbol{x}_k^\mathrm{T})\mathbf{P}_{k-1}\\
\boldsymbol{\theta}_k =& \boldsymbol{\theta}_{k-1}
 +\mathbf{K}_k(y_k-\boldsymbol{x}_k^\mathrm{T}\theta_{k-1})
\end{align*}


\section{飞行器再入动力学方程}
本节推导不考虑火星自转时的飞行器再入动力学方程,
假设飞行器只受空气动力和重力,无风干扰。

首先建立坐标系。

1. 火星赤道惯性坐标系($O-X_IY_IZ_I$):
原点$O$位于火星球心,
$OX_I$轴指向火星赤道平面与黄道平面相交节线的升交点,
$OZ_I$轴垂直于赤道平面指向北极,
$OY_I$轴与$OX_I$和$OZ_I$轴组成右手正交坐标系。
该坐标系与地球赤道惯性坐标系定义相仿。

2. 火星赤道固连坐标系($O-XYZ$):
原点$O$位于火星球心,
$OX$轴在赤道平面内指向零经度线,
$OZ$轴垂直于赤道平面指向北极,
$OY$轴与$OX$和$OZ$轴组成右手正交坐标系。
该坐标系与火星固连,
并相对火星赤道惯性坐标系以火星自转角速度$\omega$转动。

3. 飞行器位置坐标系($O-xyz$):
又称"天东北坐标系",
原点$O$位于火星球心,
$Ox$轴指向飞行器质心$M$,
$Oy$轴平行于飞行器当地水平面的东向,
$Oz$轴平行于飞行器当地水平面的北向,

4. 速度坐标系($M-x_vy_vz_v$):
原点$M$位于飞行器质心,
$Mx_v$轴平行于$Ox$轴,
$Mz_v$轴指向飞行器速度方向,
$My_v$轴与$Mx_v$和$Mz_v$轴组成右手正交坐标系。

飞行器在三维空间中的运动可以由
位置$\boldsymbol{r}(t)$和
速度$\boldsymbol{v}(t)$两个向量来表征。

%%%%%%%%%%%%%%%%%%%%%%%%%%%%%%%%%%%%%%%%%%%%%%%%%%%%%%%%%%%%%%%%
% Real
%%%%%%%%%%%%%%%%%%%%%%%%%%%%%%%%%%%%%%%%%%%%%%%%%%%%%%%%%%%%%%%%
\section{实时仿真理论}
宇宙中各天体的自转或公转周期通常以天甚至以年计算,
且轨道速度快,空间尺度大。
如果按照实际参数进行仿真,
则仿真的计算量太大而无法实时计算。
为了能够在仿真时直观地实时展示仿真结果,
需要对一些公式的单位进行换算。

万有引力公式
\begin{equation}
    \frac{\text{d}^2\vec{r}}{\text{d}t^2}=-\frac{\mu}{r^3}\vec{r} \label{eqRealGravity}
\end{equation}
中,$r$的单位是km,$t$的单位是s。
因为实际的尺度太大,一个天文单位达到了$10^8$数量级,
为了便于仿真,
需要将实际尺度中的时间和距离变换到一个合适的时空坐标系下。
首先列举一些名词解释。
\textbf{实际时间}指宇宙尺度上的实际时间;
\textbf{无单位求解器时间}指微分方程求解器中自变量的值,
没有单位,简称求解器时间,
对应的求解器速度和求解器距离也都没有单位;
\textbf{仿真展示软件消耗时间}:例如仿真展示软件运行1秒对应实际时间的1000秒,
此处的1秒指的是是仿真展示软件消耗时间,简称展示时间。

举例说明,
求解器时间的1单位时间等于实际时间的$10^4$秒,
1单位距离等于实际距离$10^6$km,即在求解器中,
\begin{equation}
    t_1=10^{-4}t,\ r_1=10^{-6}r \label{eqRealConvert}
\end{equation}
其中$t$和$r$分别为以秒和千米为单位的实际时间和实际距离,
而$t_1$和$r_1$分别表示求解器时间和求解器距离。
1个天文单位为$r=1.5*10^8$km,
则求解器中的1个天文单位为$r_1=150$。
然后可以计算出实际时间/速度/距离
和求解器时间/速度/距离之间的一些换算关系
\begin{align*}
v_1 =& \frac{\text{d}r_1}{\text{d}t_1}
 = \frac{10^{-6}\text{d}r}{10^{-4}\text{d}t} = 10^{-2}v \\
a_1 =& \frac{\text{d}}{\text{d}t_1}\frac{\text{d}r_1}{\text{d}t_1}
 = 10^2\frac{\text{d}^2r}{\text{d}t^2} \\
\mu_1(\frac{(10^{-6}\text{km})^3}{\text{kg}\cdot(10^{-4}\text{s})^2})
 =& 10^{-10}\mu(\frac{\text{km}^3}{\text{kg}\cdot \text{s}^2}) \\
-\frac{\mu_1}{r_1^2} =& -10^2\frac{\mu}{r^2} = 10^2\frac{\text{d}^2r}{\text{d}t^2} = a_1
\end{align*}
可以验证,实际中的微分方程
$$a=-\frac{\mu}{r^2}$$
变换到求解器中仍然为
$$a_1=-\frac{\mu_1}{r_1^2}$$

下面在换算公式\eqref{eqRealConvert}的基础上引入仿真展示软件消耗时间。
若仿真步长设置为0.01求解器时间,仿真展示软件每帧仿真10步,
则换算到实际时间上的步长为100秒,每帧代表1000秒,
按60帧计算,则展示时间1秒等于实际时间$6\times10^4$秒,
地球速度按30km/s算则展示时间1秒内地球走过$1.8\times10^6$km,
换算回仿真中则走过1.8展示距离。
换一种方法计算,
地球的求解器速度是$v_1=10^{-2}\times30=0.3$,
仿真展示软件每1秒仿真:
60帧$\times$10步/帧$\times$0.01单位时间/步=6个时间单位,
也可以得到展示时间1秒内地球走过1.8展示距离。
在式\eqref{eqRealConvert}的设定下,
仿真展示软件每帧仿真1步与10步对应的地球自转周期分别约为$14.4$和$1.44$秒,
分别可用于展示低轨卫星和地球同步卫星的动态运行结果。

与实时仿真不同的是实际中广泛使用的非实时仿真,
也就是在一次仿真结束后展示静态仿真结果。
此时虽然不需要考虑展示时间/速度/距离,
但求解器时间/速度/距离仍然需要考虑。
针对本文的火星气动捕获场景,
将换算公式\eqref{eqRealConvert}改为,
\begin{equation}
    t_1=10^{-4}t,\ r_1=10^{-3}r
\end{equation}
则。

如果仿真需要展示系统运动的动态过程,
则应使用实时仿真,否则应使用非实时仿真。
本文中火星气动捕获制导结果仅需要展示最终制导轨迹,
因此使用使用非实时仿真。

%%%%%%%%%%%%%%%%%%%%%%%%%%%%%%%%%%%%%%%%%%%%%%%%%%%%%%%%%%%%%%%%
% Sim
%%%%%%%%%%%%%%%%%%%%%%%%%%%%%%%%%%%%%%%%%%%%%%%%%%%%%%%%%%%%%%%%
\section{仿真}
本文使用微分方程求解器Simucpp\cite{olzhn2021}进行仿真。
下文介绍各部分设计原理。

% ////////////////////////////////////////
\subsection{被控对象建模}
可以用第2节中推导出的式\eqref{eqModelTotal}建立被控对象微分方程,
若忽略火星自转的影响,也可以建立向量形式的微分方程
\begin{equation}
    \ddot{\vec{r}} = \frac{\mu}{||\vec{r}||^3}\vec{r}+\vec{L}+\vec{D} \label{eqSimFA}
\end{equation}
其中阻力向量为
\begin{align}
    \vec{D} = -D\frac{\vec{v}}{||\vec{v}||} \label{eqSimFD}
\end{align}
升力向量$\vec{L}$与铅垂面夹角为倾侧角$\sigma$,
为了能将$\vec{L}$用其他已知向量表示,
需要构造一对与$\vec{L}$同平面的正交单位向量。
已知$\vec{v}$与$\vec{r}$张成铅垂面,
另构造一个垂直于铅垂面且在惯性系中方向向上的单位向量$\vec{n}_2$,
和另一个与$\vec{r}$同方向的单位向量$\vec{n}_1$,
则$\vec{n}_1$位于铅垂面内,
此时$\vec{L}$与$\vec{n}_1$的夹角即为倾侧角,
$\vec{n}_1$和$\vec{n}_2$即为用于表示$\vec{L}$的正交单位向量,
即
\begin{align}
    \vec{n}_2 =& \frac{\vec{r}\times\vec{v}}{||\vec{r}\times\vec{v}||} \notag\\
    \vec{L} =& \vec{n}_1\cos\sigma + \vec{n}_2\sin\sigma \label{eqSimFL}
\end{align}
微分方程式\eqref{eqSimFA}\eqref{eqSimFD}\eqref{eqSimFL}共同组成被控对象模型。
建立飞行器被控对象模型的模块框图如图\ref{figSimPlant}所示。
\begin{center}
	\includegraphics[scale=0.8]{plant.pdf}  \\
	\figcaption{被控对象模型的模块框图}\label{figSimPlant}
\end{center}
图中,被控对象的输入(控制量)为倾侧角$\sigma$,
输出为位置$\vec{r}$。
图中的两个多输入单输出函数分别为式\eqref{eqSimFLDTotal}\eqref{eqSimFATotal}
\begin{align}
    \left\{\begin{aligned}
    h =& ||\vec{r}||-R \\
    \rho =& \rho_0e^{-h/h_s} \\
    D =& \frac{1}{2}\rho||\vec{v}||^2S_{\text{ref}}C_D \\
    \vec{D} =& f_1(\vec{r},\vec{v}) = -D\frac{\vec{v}}{||\vec{v}||} \\
    L =& DC_{LD} \\
    \vec{n}_1 =& \frac{\vec{r}}{||\vec{r}||} \\
    \vec{n}_2 =& \frac{\vec{r}\times\vec{v}}{||\vec{r}\times\vec{v}||} \\
    \vec{L} =& f_2(\vec{r},\vec{v},\sigma) = L(\vec{n}_1\cos\sigma + \vec{n}_2\sin\sigma) \\
    \vec{f}_{LD} =& f_{LD}(\vec{r},\vec{v},\sigma) = \vec{L} + \vec{D}
\end{aligned}\right. \label{eqSimFLDTotal}
\end{align}
\begin{equation}
    \ddot{\vec{r}} = f_A(\vec{r},\vec{f}_{LD}) = \frac{\mu}{||\vec{r}||^3}\vec{r}+\frac{\vec{f}_{LD}}{m} \label{eqSimFATotal}
\end{equation}
式中$R$为火星半径,$C_{LD}=C_L/C_D$为升阻比。

% ////////////////////////////////////////
\subsection{简化模型仿真结果}
给定进入点轨道初值和控制目标如下。
\begin{center}\begin{tabular}{lll}
    \toprule
    条件 & 初值 & 目标 \\
    \midrule
    进入点速度(km/s) & 5.8 \\
    进入点高度(km) & 125 \\
    进入航迹角$\gamma$(rad) & 0.192(11$^\circ$) \\
    偏心率 & 1.75 & 0.215 \\
    轨道倾角$i$(rad) & 0.524(30$^\circ$) & 0.524 \\
    升交点赤经(rad) & 0 & \\
    近拱点幅角(rad) & 0 & \\
    远拱点高度(km) & & 2250 \\
    近拱点高度(km) & & 250 \\
    \bottomrule
\end{tabular}\end{center}
设定参数中,偏心率可由进入点速度、进入点高度、进入航迹角得出,
由第\ref{secFormElement}小节公式可验证设定参数基本正确。
本文仿真暂不考虑轨道倾角。

首先展示不使用制导律,控制量输出常值的仿真结果,取$\sigma=1$。
3D仿真结果如图\ref{figSim1View}所示。
\begin{center}
	\includegraphics[scale=0.2]{sim1Verticalview.png}  \\
	\includegraphics[scale=0.2]{sim1Sideview.png}  \\
	\figcaption{确定轨迹3D仿真结果}\label{figSim1View}
\end{center}
图\ref{figSim1View}中的两幅图分别为正等轴测图和侧视图。
轨道高度变化曲线如图\ref{figSim1Height}所示。
横坐标为时间,单位为0.1秒,截取了前60秒。
\begin{center}
	\includegraphics[scale=0.6]{sim1Height.pdf} \\
	\figcaption{轨道高度变化曲线}\label{figSim1Height}
\end{center}

% ////////////////////////////////////////
\subsection{制导律仿真结果}
制导律与各种仿真参数见程序。
期望远拱点半径与制导律作用下的实际远拱点半径均为$5652.5$km。
图\ref{figSim2State}所示分别为高度、倾侧角、侧向速度的变化曲线,
原始数据的单位分别为km、rad、$10^{-4}$km/s。
此时未加侧向制导的约束。
\begin{center}
	\includegraphics[scale=0.6]{Sim2State.pdf} \\
	\figcaption{未加侧向制导约束时的多种状态曲线}\label{figSim2State}
\end{center}
加入侧向制导约束后的结果如图\ref{figSim2State2}所示。
\begin{center}
	\includegraphics[scale=0.6]{Sim2State2.pdf} \\
	\figcaption{加入侧向制导约束后的多种状态曲线}\label{figSim2State2}
\end{center}
3D仿真结果如图\ref{figSim2View}所示,两幅图分别为俯视图和侧视图。
\begin{center}
	\includegraphics[scale=0.2]{sim2Verticalview.png}  \\
	\includegraphics[scale=0.2]{sim2Sideview.png}  \\
	\figcaption{制导律3D仿真结果}\label{figSim2View}
\end{center}
对精确模型的飞行器飞出大气层后的完整轨迹,使用解析法计算过程较复杂,
所以大气层内外全程采用数值解,计算量较大,
而且数值解难以计算飞行器在什么时候飞到了远拱点。
数值解步长为$10^{-3}$,一个步长代表$0.01$秒,
大气层内每100个步长、大气层外每1000个步长记录一次坐标。
从飞出大气层时开始计算的4000秒时飞行器的向径为$5650.39$km,
也就是说图\ref{figSim2View}中的轨迹末端并不是严格的远拱点。

以后还将加入大气变化等不确定性因素,
时间有限暂不展开。

\section{结\quad 论}
本文。

\bibliographystyle{stylebib}
\bibliography{reference}


\end{multicols}
\end{document}
